% Conclusion

\chapter{Conclusion}

%\section{Context}

La découverte du boson de Higgs en 2012 n'est pas la conclusion du \MS, loin de là. 
Il faut à présent comprendre cette nouvelle particule et découvrir ce qu'il existe à plus haute énergie.
Car pour le moment, le \MS n'explique pas tout (secteur noir), il reste des déviations entre prédictions et expériences (moment magnétique anormal du muon), mais aussi la séparation totale avec la relativité.

C'est pourquoi il faut continuer la recherche.
Et les expériences de type ILC et FCC peuvent nous aider à le perfectionner. 
Mais pour le moment, les chantiers associés sont toujours en attente. \\

%\section{Stage}

Dans le cadre de mon stage, j'ai donc travaillé au sein de la collaboration CALICE sur le projet SDHCAL, qui est un détecteur candidat pour les projets ILC et FCC. 

Mon objectif était double, comprendre et améliorer le code déjà existant afin de les rendre plus facile à l'emploi. Puis de les modifier pour qu'il puisse s'adapter du projet ILC au projet FCC.

Pour la première partie, j'ai développé des outils de factorisation de code, complété la documentation et développé de nombreux programmes d'exécution et de tests des résultats.
Malheureusement, la second partie n'a pas pu être la terminer dans les temps (ce qui avait été anticipé avant le début de mon stage). 
Mais j'ai rédigé énormément de documentation pour expliquer ce que j'ai fait et pour permettre à mon successeur de ne pas s'attarder sur les mêmes problèmes que moi.


\section*{Remerciements}

Je souhaite d'abord remercier Gérald \textsc{Grenier} pour m'avoir donner la chance de montrer ce que je peux faire. Et aussi pour son encadrement, son accompagnement et son temps.

Je souhaite plus largement remercier mon équipe, Gérald \textsc{Grenier}, Imad \textsc{Laktineh} et Clément \textsc{Devanne}, pour l'atmosphère positive, détendue et stimulante.

Et plus largement, les employés de l'IP2I pour leur gentillesse et leur accueil.\\


De plus, au cours de mon stage, j'ai pu m'impliquer dans la vie du laboratoire, en participant au stand tenu par l'IP2I à la \textit{Geek and Japon Touch}, organisé par Stéphanie \textsc{Beauceron} (IP2I, CNRS, CMS).

Durant ce week-end, avec 2 autres chercheuses (non physiciennes), Florence \textsc{Boyer} et Liliane \textsc{De Araujo}, on a tenu un débat sur le film \textit{Don't look up : Déni Cosmique} de Adam \textsc{McKay}, sur la crédibilité du discours scientifiques.

Ensuite sur le stand, j'ai pu expliquer les bases scientifiques et des recherches menées par le CNRS, le CERN, et Virgo au près du grand publique. 
Et comme je n'ai pas fait ça toute seule, je tiens à remercier tous les autres participant pour ce week-end, soient de nombreux chercheurs et doctorants de l'IP2I, Stéphanie \textsc{Beauceron}, Nazila \textsc{Mahmoudi}, Bastien \textsc{Voirin}, Brigitte \textsc{Cheynis}, Grégoire \textsc{Pierra} et aussi Jérôme \textsc{Degallaix} du LMA et Benjamin \textsc{Blanco} un stagière de M1 de CMS.

J'ai aussi animé le stand "visualisation de la gravité" de l'association Créativ' Sciences dont l'objectif est d'expliquer les principes de base de la gravité en 2D avec un drap tendu.
