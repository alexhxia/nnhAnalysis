\documentclass[10pt,a4paper]{report}

\usepackage[utf8]{inputenc}
\usepackage[french]{babel}
\usepackage[T1]{fontenc}
\usepackage{amsmath}
\usepackage{amsfonts}
\usepackage{amssymb}
\usepackage{graphicx}
\usepackage{lmodern}
\usepackage{hyperref}
\usepackage{url}
\usepackage{xspace}
\usepackage{tikz}
\usepackage{listings}
\usepackage{caption}
\usepackage{subcaption}
\usepackage{enumitem}
%\usepackage{feynmp}	
%\usepackage{tikz-feynman}

\usepackage[left=2cm,right=2cm,top=2cm,bottom=2cm]{geometry}
%%%%%%%%%%%%%%%%%%%%%%%%%%%%% Page de garde %%%%%%%%%%%%%%%%%%%%%%%%%%%%%
\author{Alexia \textsc{HOCINE}}

\title{
	Détecteur SDHCAL pour le signal  $ e^{+} e^{-} \longrightarrow \nu \nu h $ :\\
	Optimisation et Adaptation de l'analyse de données\\pour le Projet
 FCC 
}
%Stage M2 Physique, parcours SUBA\\Université de Claude Bernard Lyon 1
\date{Juillet 2022}

%%%%%%%%%%%%%%%%%%%%%%%%%%%%% Raccourçi %%%%%%%%%%%%%%%%%%%%%%%%%%%%%

% raccourci français
\newcommand{\cad}{c'est-à-dire\xspace}
\newcommand{\qqs}{quelque soit\xspace}
\newcommand{\MS}{Modèle Standard\xspace}


% nom informatique
\newcommand{\ROOT}{\texttt{ROOT}\xspace}
\newcommand{\SLCIO}{\texttt{SLCIO}\xspace}
\newcommand{\iLCSoft}{\texttt{iLCSoft}\xspace}
\newcommand{\LCIO}{\texttt{LCIO}\xspace}
\newcommand{\Marlin}{\texttt{Marlin}\xspace}
\newcommand{\Gaudi}{\texttt{Gaudi}\xspace}
\newcommand{\FCC}{\texttt{FCC}\xspace}
\newcommand{\EDMhep}{\texttt{EDM4hep}\xspace}

\newcommand{\nnhAnalysis}{\texttt{nnhAnalysis}\xspace}

\newcommand{\original}{\texttt{original}\xspace}
\newcommand{\ilcsoft}{\texttt{ilcsoft}\xspace}
\newcommand{\fcc}{\texttt{fcc}\xspace}

\newcommand{\minidstmarker}{\texttt{miniDSTMaker}\xspace}
\newcommand{\convert}{\texttt{convert}\xspace}
\newcommand{\processor}{\texttt{processor}\xspace}
\newcommand{\analysis}{\texttt{analysis}\xspace}


% nom de particules
\newcommand{\particle}[1]{$\texttt{#1}$}

\newcommand{\bbar}{\overline{b}}
\newcommand{\Wstar}{W^{\star}}

\newcommand{\electron}{e^{+}}
\newcommand{\positron}{e^{-}}

\newcommand{\nnh}{\nu \nu h}

% processus courants
\newcommand{\bb}{$\mathrm{b\bbar}$\xspace}
\newcommand{\WW}{$\mathrm{W\Wstar}$\xspace}

% physique
\newcommand{\GeV}{\mathrm{GeV}\xspace}
\newcommand{\TeV}{\mathrm{TeV}\xspace}
\newcommand{\MeV}{\mathrm{MeV}\xspace}

\newcommand{\m}[1]{m_\mathrm{#1}}
\newcommand{\mH}{m_\mathrm{H}}
\newcommand{\mW}{m_\mathrm{W}}

\newcommand{\vmH}{125,25}
\newcommand{\vmW}{80,377}
\newcommand{\vs}{250}
\newcommand{\vGF}{1,166\,378\,8 \times 10^{-5}}
\newcommand{\vhc}{0,389\,379\,365\,648} % (hbar c)^2


\newcommand{\GF}{G_\mathrm{F}}
\newcommand{\gHWW}{g_\mathrm{HWW}}

% LaTeX

\newcommand{\doToList}[1]{{\color{red}#1}}
\newcommand{\Figure}[1]{[\textsc{Figure~#1}]}


%%%%%%%%%%%%%%%%%%%%%%%%%%%%% corps du document %%%%%%%%%%%%%%%%%%%%%%%%%%%%%

\begin{document}

% Page de Garde

\begin{titlepage}
	\begin{center}
		
    		\vspace*{3cm}

    		\LARGE
    		\textbf{Détecteur SDHCAL pour le signal  $ e^{+} e^{-} \longrightarrow \nu \nu h $ :\\Optimisation et Adaptation de l'analyse de données\\pour le Projet FCC}

    		\vspace{1.5cm}

    		Alexia \textsc{HOCINE}\\
    		\vspace{0.4cm}
    		\large 
    			Étudiante en M2 Physique SUBA à l'UCBL1
    		
		\vspace{1.5cm}
		\Large
		Supervisé par Gérald \textsc{Grenier}\\
    		\vspace{0.4cm}
    		\large 
    			Maître de Conférence - UCBL1\\
    			Enseignant-Chercheur - IP2I, CNRS, IN2P3\\
    			Membre des collaborations CALICE, CMS et ILD\\
    			Corresponsable du groupe SDHCAL au sein de la collaboration CALICE

    		\vfill

		\includegraphics[width=5cm]{../img/Logo_IP2I.png}
 		\includegraphics[width=5cm]{../img/UdL-logo.png}

		\vspace{3cm} 

    		Rapport de Stage - Master 2 Physique SUBA\\
    		Université de Claude Bernard Lyon 1 \\    		

    		\vspace{1.5cm}
    		
    		2021-2022
    
 		
	\end{center}

\end{titlepage}

\tableofcontents

\chapter{Introduction}

\section{Objectifs physiques}

\subsection{Collisions}
Au cours, de ce stage, je me concentrerais sur les collisions de type \texttt{nnh} pour neutrino-neutrino-higgs

\section{SDHCAL (Semi-Digital Hadronic CALorimeter)}

tests en Septembre

\section{iLCSoft}

\section{FCC}

\section{Présentation \& Objectif du Stage}


\chapter{\texttt{ilcsoft}}

\section{Projet \nnhAnalysis}

\section{Programme \processor}

\subsubsection{Données}
Initialement, on m'a mis à disposition des fichiers \SLCIO rangés par processus dans 66 dossiers (Figure~\ref{listeProcessus}).

\begin{figure}[h!]
	%\includegraphics[width=\textwidth]{../img/listeProcessus.png} 
	\caption{Les noms des dossiers qui correspondent aux numéros de processus}
	\label{listeProcessus}
	
\end{figure}

\paragraph{Numéro des processus ???}

%\paragraph{Résultats mis à disposition ???}

\subsubsection{Méthodes}

On cherche à convertir ces fichiers \SLCIO en arbre \ROOT par processus.

\subsubsection{Résultats}

%\paragraph{Des fichiers \ROOT :}
Chaque dossier de fichier de donnée \SLCIO produira un fichier \ROOT en sortie, \cad que l'on obtiendra un arbre \ROOT par processus.


\subsubsection{Interprétation}

\section{Programme \analysis}

\subsubsection{Données}

On récupère les fichiers \ROOT du programme \processor précédent. 

$ hadd $ qui va créer le fichier DATA.root

\subsubsection{Méthodes}

\paragraph{\texttt{BDT}}

Entrainement

\paragraph{L'analyse}



\subsubsection{Résultats}

\paragraph{Vérification des résultats}
Comparaison entre les différents séries d'analyse, basée sur les même fichiers \ROOT, mais un autre entraînement de BDT.

\subsubsection{Interprétation}
\chapter{Programme \texttt{FCC}}

\section{Présentation du Projet FCC}

Le FCC (Futur Collisionneur Circulaire) est le projet du CERN pour remplacer 
leur collisionneur actuelle, le LHC (Large Hadronic Collider). 
Dont la fin de l'exploitation est prévu en 2040 \cite{cern:fcc}.
On prévoit un anneau de 100 km, contre 27 km pour le LEP et le LHC 
(comme montrer Figure~\ref{fcc:img}).
Ce qui devrait nous permettra d'atteindre une énergie de 100 $\TeV$ contre 13 $\TeV$
actuellement pour le LHC.

L'objectif est la rechercher d'une nouvelle physique par la mise en évidence de déviation avec le modèle standard. Et plus particulièrement, en augmentant la statistique sur le boson de Higgs, découvert avec le collisionneur actuelle, afin de mieux comprendre sa physique.

\begin{figure}[!ht]
    \centering
    \includegraphics[width=\textwidth]{../img/FCC.jpg}
    \caption{https://cds.cern.ch/images/OPEN-PHO-ACCEL-2019-001-2}
    \label{fcc:img}
\end{figure}

Une fois encore, le détecteur SDHCAL candidate pour ce projet et pourrait être y être installer.

\section{Développement Numérique}

Mon objectif est d'adapter les codes développés précédemment par Guillaume 
\textsc{Garillot} du projet ILC au projet FCC qui n'utilise pas les mêmes suites logiciels.\\


\subsection{\convert}

Dans un premier temps, je dois convertir les fichiers \SLCIO en fichiers \ROOT en utilisant le programme libre \texttt{k4MarlinWrapper} du projet de \texttt{key4hep}\footnote{\url{https://github.com/key4hep/k4MarlinWrapper}}. 
Il s'agit d'une collaboration entre chercheurs du CERN dont l'objectif est de développer des outils pour le projet FCC, y compris des utils de conversion entre \LCIO (\iLCSoft) et \EDMhep (\FCC).%, en utilisant les algorithmes \texttt{Gaudi}
\\

Cette partie a été laborieux, car le programme \texttt{k4MarlinWrapper} est en cours de développement. 
J'ai du faire remonter de très nombreux problèmes en parallèle de mon travail. 
En effet, à chaque mise à jour, le programme ne fonctionnait plus, ce qui m'a fait perdre beaucoup de temps en essayant de comprendre le problème, le signaler et le corriger quand je le comprenais, ou patienter pour que quelqu'un d'autre le corrige. 

\subsection{\processor}

Comme les fichiers d'entrées sont à présent des fichiers \ROOT, il faut adapter cette partie pour que les fichiers de sortie soit les mêmes ou au moins équivalent.

\subsection{\analysis}

Cette partie reste inchangé par rapport à \ilcsoft.
% Outils Numériques


\chapter{Outils Numériques}

Pour utiliser et comparer les résultats obtenus facilement, j'ai développé de très nombreux programmes supplémentaires afin de pouvoir automatiser leur utilisation, mais aussi de les exécuter sur un serveur distant. 

\section{Répertoire \texttt{script}}

Dans ce dossier\footnote{\url{https://github.com/alexhxia/nnhAnalysis/tree/main/script}}, j'ai développé des outils d'automatisation de l'exécution des programmes. En effet, avant pour lancer un \processor ou une \analysis il y avait de très nombreuses commandes à taper au terminal et parfois incompatible entre elles.
J'ai donc développé 6 programmes principaux d'automatisation des programmes :

\begin{description}
	
	\item[\texttt{nnh}]	programme générale qui permet l'exécution de tout le programme (\convert, \processor et \analysis) et plusieurs fois.
	\begin{description}

	\item[\texttt{nnhConvert}] permet de convertir les fichiers \SLCIO en fichiers \ROOT.
	
	\item[\texttt{nnhProcessor}] d'exécuter tout le programme \processor.
	
	\item[\texttt{nnhAnalysis}] exécute tout le programme \analysis.
	
	\begin{description}
		
		\item[\texttt{prepareBDT}] exécute la préparation de la BDT.
		
		\item[\texttt{launchBDT}] lance la BDT \footnote{Le programme \texttt{launchBDT} est incompatible avec les autres, ce qui nécessite de le lancer dans un terminal enfant.}.
		
	\end{description}
	\end{description}
	
\end{description}

Chacun de ces programmes peuvent être lancé ensemble mais aussi séparément. Ce qui permet de gagner énormément en temps d'utilisation mais aussi de pouvoir personnaliser selon les besoins de l'utilisateur.

\section{Répertoire \texttt{test}}

Pour tester les résultats obtenus, j'ai développé 4 programmes en \texttt{python}.  Et de tester s'ils sont compatibles entre eux.

\begin{figure}[h!]
	\center
	\begin{tabular}{| c | c | c |}
		\hline
			\texttt{•} & \texttt{Processus} & \texttt{Analysis} \\
		\hline
			\texttt{Completed} & \texttt{testProcessorCompleted.py} & \texttt{testAnalysisCompleted.py} \\
		\hline
			\texttt{Same} & \texttt{testProcessorSame.py} & \texttt{testAnalysisSame.py} \\
		\hline
	\end{tabular}
	\caption{Tableau récapitulatif des fonctions de tests}
\end{figure}

\subsection{Programmes \texttt{testXxCompleted.py}}

L'objectif de ce type de programme est de tester si tous les fichiers qui aurait du être créer, l'ont bien été. Ces programmes sont très rapides, ne prennent que en quelques secondes.

Ce qui permet, en cas de problèmes de relancer juste la partie qui n'est pas terminé et pas tout le programme.

\subsubsection{Programmes \texttt{testProcessorCompleted.py}}

Un \processor est complet si tous les dossiers du répertoire d'entré (\Figure{\ref{data:list}}) ont bien généré un fichier \ROOT.

\subsubsection{Programmes \texttt{testAnalysisCompleted.py}}

Une \analysis est complète si elle contient 13 fichiers, car :
\begin{description}
	\item[hadd :] 1 fichier \verb|data.root|
	\item[prepareBDT :] 4 fichiers (2 polarisations $\times$ 2 canaux) \verb|split_XX.root| 
	\item[launchBDT :] 2 fichiers (2 canaux) \verb|model_XX.joblib|, \verb|scores_XX.root|, \verb|bestSelection_XX.root|, \verb|stats_XX.json|
\end{description}

\subsection{Programmes \texttt{testXxSame.py}}

Ce type de programme\footnote{\url{https://github.com/alexhxia/nnhAnalysis/tree/main/test}
} va prendre quelques dizaines de minutes à une heure pour s'exécuter, car il va comparer tous les arbres \ROOT des différents fichiers pour s'assurer que 2 fichiers sont identiques ou au moins compatibles. Cette comparaison se ferait avec la fonction de Kolmogorov présente, là encore, dans l'API du CERN, qui compare 2 histogrammes et retourne un flottant sur leur compatibilité. Dans le chapitre \original, j'ai explicité que les programmes qui donnaient des résultats toujours identiques et ceux qui devaient avoir des résultats équivalents. 

\subsubsection{Programmes \texttt{testProcessorSame.py}}

Toutes les exécutions de \processor doivent donner des résultats identiques, surtout au sein du même projet. Et j'ai pu le confirmer en comparant plusieurs résultats obtenus de différentes exécutions. 

Et lorsque j'ai apporté des corrections au programme, quand je suis passée de \original à \ilcsoft, cela m'a permis de vérifier que mon code obtenait bien le même résultat. De même, de \ilcsoft à \fcc.

\subsubsection{Programmes \texttt{testAnalysisSame.py}}

La comparaison entre 2 exécutions du programme \analysis est plus compliquée, puisque la BDT utilise des nombres aléatoires, ce qui ne permet pas d'obtenir des résultats identiques.

Certains fichiers doivent rester identiques comme les fichiers \verb|data.root| ou \verb|split_XX.root|. Certains sont complètement différents comme les fichiers \verb|model_XX.joblib| ou \verb|bestSelection_XX.root|, car ils sont liés à l'entraînement de la BDT. Mais à la fin les fichiers \verb|stat_XX.json| doivent être statistiquement compatibles.

\subsection{Répertoire \texttt{result}}

Dans ce répertoire, je place les fichiers de sorti des programmes \texttt{test}. 
En effet, quand on exécute un programme de test, l'utilisateur a la possibilité d'obtenir les résultats sous la forme d'un fichier \texttt{JSON}. 
Ce qui permet d'en garder une trace et de ne pas refaire un test déjà effectué.\\

\subsection{Mes résultats}

Je peux donc confirmer que mes programmes s'exécutent complètement, correctement et avec des écarts statistiques faibles aux finals (générés par l'utilisation de BDT).

%%%%%%%%%%%%%%%%%%%%%%%%%%%%%

\chapter{Résultats Physiques}

\section{Résultats Numériques}

Les résultats de l'étude statistique sont dans les fichiers de type \verb|stats_XX.json|, et sont triés suivant 10 catégories. 

\begin{figure}[!ht]
	\centering
	\begin{subfigure}[b]{0.45\textwidth}
		\begin{description}
			\item[0] \verb|Signal|
			\item[1] \verb|Other higgs->WW*|
			\item[2] \verb|Other higgs|
			\item[5] \verb|2 fermions leptonic|
			\item[6] \verb|2 fermions hadronic|
			\item[8] \verb|4 fermions hadronic|
			\item[9] \verb|4 fermions semileptonic|
			\item[7] \verb|4 fermions leptonic|
		\end{description}
		\label{stats:results:WW}
		\caption{Pour le canaux \WW}
	\end{subfigure}
     \hfill
	\begin{subfigure}[b]{0.45\textwidth}
		\begin{description}
			\item[0] \verb|Signal|
			\item[3] \verb|Other higgs->bb|
			\item[4] \verb|Other higgs|
			\item[5] \verb|2 fermions leptonic|
			\item[6] \verb|2 fermions hadronic|
			\item[8] \verb|4 fermions hadronic|
			\item[9] \verb|4 fermions semileptonic|
			\item[7] \verb|4 fermions leptonic|
		\end{description}
		\caption{Pour le canaux \bb}
		\label{stats:results:bb}
	\end{subfigure}
	\label{stats:results}
	\caption{Les numéros et les noms des différentes catégories d'\analysis pour l'étude statistique}
\end{figure}

Et dans chaque catégorie, plusieurs variables sont calculées \Figure{\ref{stats:results:vrb}}.

\begin{figure}[!ht]
	\centering
	\begin{description}
		\item[\texttt{name}] nom du type de l'évènement, 
				cité dans \Figure{\ref{stats:results}}
		\item[\texttt{stat}] nombre total d'évènement mesuré
		\item[\texttt{sum}] somme des poids de chaque évènement (signal sur bruit)
		
		\item[\texttt{statPreSel}] nombre d'évènement pré-sélectionnés par la BDT
		\item[\texttt{effPreSel}] taux de pré-sélection
		\item[\texttt{sumPreSel}] somme des poids des évènements pré-sélectionnés 
		
		\item[\texttt{effSel}] taux de sélection
		\item[\texttt{sumSel}] somme des poids des évènements sélectionnés
	\end{description}
	\caption{Variables statistiques de chacune des catégories précédentes \Figure{\ref{stats:results}}}
	\label{stats:results:vrb}
\end{figure}

\subsection{Partie \original}

Regardons 2 résultats d'\analysis différents pour \bb, dans le cas d'un signal \Figure{\ref{stats:results:img}} :

\begin{figure}[!ht]
	\centering
	\begin{subfigure}[b]{0.45\textwidth}
		\includegraphics[width=\textwidth]{../img/stats_bb_original_run1_run1.png} 
		\label{stats:results:bb:original:1:1}
		\caption{Canal \bb, branche \original, run 1}
	\end{subfigure}
     \hfill
	\begin{subfigure}[b]{0.45\textwidth}
		\includegraphics[width=\textwidth]{../img/stats_bb_original_run2_run1.png}
		\caption{Canal \bb, branche \original, run 2}
		\label{stats:results:bb:original:2:1}
	\end{subfigure}
	\label{stats:results:img}
	\caption{Les numéros et les noms des différentes catégories d'\analysis pour l'étude statistique}
\end{figure}

On constate un écart initial sur le nombre d'évènement de moins de 0.01\%, puis une diminution des évènements pré-sélectionnés de 17,6\% pour le \textit{run 1} contre 17,5\% pour le \textit{run 2}. Et une diminution de 59,6\% et 60,0\% pour les évènements sélectionnés par la BDT. 

L'autre constat est qu'on subit une grosse diminution entre les variables \texttt{sum} et \texttt{effSel}, de 88,3\% et 88,5\%, ce qui traduit une baisse importante de luminosité induit par le signal, une fois débarrassée du bruit de fond. Mais une fois encore ces résultats sont cohérents. \\

Donc même si les résultats sont proches, ils ne sont pas identiques puisque la BDT utilise des nombres aléatoires dans sa prise de décision sur le type d'évènement. Mais le programme reste robuste, et que l'utilisation d'une BDT n'est pas une solution idéale puisqu'elle induise une incertitude numérique supplémentaire.

%
\section{Physique du Higgs}

\subsection{Section efficace}

\subsubsection{Quelques constantes}
Avant de commencer, je vais rappeler certaines constantes utiles pour la suite tirée du PDG, \textit{Particle Data Group} 2022\cite{Workman:2022ynf} :
\begin{description}

	\item[$\gHWW$] : couplage du boson de Higgs et du boson W

	\item[$\GF$] : constante de couplage de Fermi
	\footnote{\url{https://pdg.lbl.gov/2021/tables/contents_tables.html}}
	$$ \frac{G_\mathrm{F}}{\left(\hbar c\right)^3} = 1,166\,378\,7(6) \times 10^{-5}\, \GeV^{-2} $$
	$$ (\hbar c)^2 = \vhc \, \GeV^2 \, \mathrm{mbarn} = \vhc \times 10^{-31} \, \mathrm{m}.\GeV^2 $$
	
	\item[$\mH$] : masse du boson de Higgs
	\footnote{\url{https://pdglive.lbl.gov/Particle.action?node=S126&init=0}}
	$$ m_\mathrm{H_0} = 125,25 \pm 0,17 \, \GeV $$
		
	\item[$m_W$] : masse du boson W
	\footnote{\url{https://pdglive.lbl.gov/Particle.action?node=S043&init=0}}
	$$ m_\mathrm{W} = 80,377 \pm 0,012 \, \GeV $$
	
	\item[$s$] : variable de Mandelstam
	$$ s = \left(p_1 + p_2\right)^2 = \left(p_{e} + p_{p}\right)^2 $$
	Mes données correspondent aux simulations à : 
	$$ \sqrt{s} = 250\, \GeV $$

		
\end{description}

%%%%%%%%%%%%%%%%%%%%%%%%%%%%%%%%%%%

\subsubsection{Fusion $W\Wstar$ \cite{desy}}

Comme on est à très haute énergie, on peut approximer que $ \sqrt{s} >> 2 \mW $ :

\begin{equation}
	\sigma_\mathrm{WW-fusion} \longrightarrow 
		\frac{\gHWW^2 \, \GF^2}{32 \, \pi^3}
		\left[
			\left(1 + \frac{\mH^2}{s}\right) \log\left(\frac{s}{\mH^2}\right)
			- 2 \left(1 - \frac{\mH^2}{s}\right)
		\right]
\end{equation}


\begin{align*}
	\sigma_\mathrm{WW-fusion}
		&\approx\frac{\gHWW^2 \, \GF^2}{32 \, \pi^3}
		\left[
			\left(1 + \frac{\mH^2}{s}\right) \log\left(\frac{s}{\mH^2}\right)
			- 2 \left(1 - \frac{\mH^2}{s}\right)
		\right]\\
		&=\gHWW^2\frac{(\hbar c)^6}{32 \, \pi^3}
		\left[\frac{\GF}{(\hbar c)^3}\right]^2
		\left[
			\left(1 + \frac{\mH^2}{s}\right) \log\left(\frac{s}{\mH^2}\right)
			- 2 \left(1 - \frac{\mH^2}{s}\right)
		\right]\\
		&=\gHWW^2\frac{(\vhc)^3}{32 \, \pi^3}
		\left[\vGF\right]^2
		\left[
			\left(1 + \frac{\vmH^2}{\vs^2}\right) \log\left(\frac{\vs^2}{\vmH^2}\right)
			- 2 \left(1 - \frac{\vmH^2}{\vs^2}\right)
		\right]\\
		&\approx \gHWW^2 \times (-6,0466389\times 10^{-15})
\end{align*}

%%%%%%%%%%%%%%%%%%%%%%%%%%%%%%%%%%%

\subsubsection{La largeur partielle pour $W\Wstar$ \cite{desy}}

\begin{equation}
	\Gamma\left(H\longrightarrow WW\right) = 
		\frac{\gHWW^2 \, \mH^3}{64 \, \pi \, \mW^2}
		\left(1 - \frac{4 \mW^2}{\mH^2} + \frac{12 \mW^4}{\mH^4}\right)
		\left(1 - \frac{4 \mW^2}{\mH^2}\right)^{1/2}
\end{equation}

\begin{align*}
	\Gamma\left(H\longrightarrow WW\right) 
	&= \gHWW^2
		\frac{\vmH^3}{64 \, \pi \, \vmW^2}
		\left(1 - \frac{4\times \vmW^2}{\vmH^2} + \frac{12\times \vmW^4}{\vmH^4}\right)
		\left(1 - \frac{4\times \vmW^2}{\vmH^2}\right)^{1/2}\\
	&= \gHWW^2 
\end{align*}

%%%%%%%%%%%%%%%%%%%%%%%%%%%%%%%%%%%

\subsubsection{La section efficace de production du Higgs}

\begin{equation}
	\sigma_\mathrm{production} = \sigma_\mathrm{WW-fusion} \times \Gamma\left(H\longrightarrow WW\right)
\end{equation}

\begin{align*}
	\sigma 
		&= \frac{\gHWW^2 \, \GF^2}{32 \, \pi^3}
		\left[
			\left(1 + \frac{\mH^2}{s}\right) \log\left(\frac{s}{\mH^2}\right)
			- 2 \left(1 - \frac{\mH^2}{s}\right)
		\right]
		\times 
		\frac{\gHWW^2 \, \mH^3}{64 \, \pi \, \mW^2}
		\left(1 - \frac{4 \mW^2}{\mH^2} + \frac{12 \mW^4}{\mH^4}\right)
		\left(1 - \frac{4 \mW^2}{\mH^2}\right)^{1/2} \\
	\sigma &= \gHWW^4 \frac{\GF^2}{32 \, \pi^4} \frac{\mH^3}{64 \, \mW^2}
		\left[
			\left(1 + \frac{\mH^2}{s}\right) \log\left(\frac{s}{\mH^2}\right)
			- 2 \left(1 - \frac{\mH^2}{s}\right)
		\right]
		\left(1 - \frac{4 \mW^2}{\mH^2} + \frac{12 \mW^4}{\mH^4}\right)
		\left(1 - \frac{4 \mW^2}{\mH^2}\right)^{1/2} \\
	\sigma \propto \gHWW^4
\end{align*}

%%%%%%%%%%%%%%%%%%%%%%%%%%%%%%%%%%%

\subsection{Nombre de Higgs attendu}

Le nombre d'évènement est :
\begin{equation}
^	\mathcal{N}_\mathrm{event} 
	= \mathcal{N}_\mathrm{signal} 
	+ \mathcal{N}_\mathrm{background}
\end{equation}

Or le nombre d'évènement détecté est :
\begin{equation}
	\mathcal{N}_\mathrm{detect} 
	= \mathcal{N}_\mathrm{event} \pm \sigma(\mathcal{N}_\mathrm{event})
	= \mathcal{N}_\mathrm{event} \pm \sqrt{\mathcal{N}_\mathrm{event}}
\end{equation}

Et le signal est :
\begin{equation}
	\mathcal{N}_\mathrm{signal} 
	= \mathcal{N}_\mathrm{detect} 
	- \mathcal{N}_\mathrm{background}
\end{equation}

Donc le signal est :
\begin{equation}
	\mathcal{N}_\mathrm{detect} 
	- \mathcal{N}_\mathrm{background} 
	- \sqrt{\mathcal{N}_\mathrm{évènement}}
	\leq \mathcal{N}_\mathrm{signal} \leq
	\mathcal{N}_\mathrm{detect} 
	- \mathcal{N}_\mathrm{background} 
	+ \sqrt{\mathcal{N}_\mathrm{event}}
\end{equation}


%%%%%%%%%%%%%%%%%%%%%%%%%%%%%

% Conclusion

\chapter{Conclusion}

%\section{Context}

La découverte du boson de Higgs en 2012 n'est pas la conclusion du \MS, loin de là. 
Il faut à présent comprendre cette nouvelle particule et découvrir ce qu'il existe à plus haute énergie.
Car pour le moment, le \MS n'explique pas tout (secteur noir), il reste des déviations entre prédictions et expériences (moment magnétique anormal du muon), mais aussi la séparation totale avec la relativité.

C'est pourquoi il faut continuer la recherche.
Et les expériences de type ILC et FCC peuvent nous aider à le perfectionner. 
Mais pour le moment, les chantiers associés sont toujours en attente. \\

%\section{Stage}

Dans le cadre de mon stage, j'ai donc travaillé au sein de la collaboration CALICE sur le projet SDHCAL, qui est un détecteur candidat pour les projets ILC et FCC. 

Mon objectif était double, comprendre et améliorer le code déjà existant afin de les rendre plus facile à l'emploi. Puis de les modifier pour qu'il puisse s'adapter du projet ILC au projet FCC.

Pour la première partie, j'ai développé des outils de factorisation de code, complété la documentation et développé de nombreux programmes d'exécution et de tests des résultats.
Malheureusement, la second partie n'a pas pu être la terminer dans les temps (ce qui avait été anticipé avant le début de mon stage). 
Mais j'ai rédigé énormément de documentation pour expliquer ce que j'ai fait et pour permettre à mon successeur de ne pas s'attarder sur les mêmes problèmes que moi.


\section*{Remerciements}

Je souhaite d'abord remercier Gérald \textsc{Grenier} pour m'avoir donner la chance de montrer ce que je peux faire. Et aussi pour son encadrement, son accompagnement et son temps.

Je souhaite plus largement remercier mon équipe, Gérald \textsc{Grenier}, Imad \textsc{Laktineh} et Clément \textsc{Devanne}, pour l'atmosphère positive, détendue et stimulante.

Et plus largement, les employés de l'IP2I pour leur gentillesse et leur accueil.\\


De plus, au cours de mon stage, j'ai pu m'impliquer dans la vie du laboratoire, en participant au stand tenu par l'IP2I à la \textit{Geek and Japon Touch}, organisé par Stéphanie \textsc{Beauceron} (IP2I, CNRS, CMS).

Durant ce week-end, avec 2 autres chercheuses (non physiciennes), Florence \textsc{Boyer} et Liliane \textsc{De Araujo}, on a tenu un débat sur le film \textit{Don't look up : Déni Cosmique} de Adam \textsc{McKay}, sur la crédibilité du discours scientifiques.

Ensuite sur le stand, j'ai pu expliquer les bases scientifiques et des recherches menées par le CNRS, le CERN, et Virgo au près du grand publique. 
Et comme je n'ai pas fait ça toute seule, je tiens à remercier tous les autres participant pour ce week-end, soient de nombreux chercheurs et doctorants de l'IP2I, Stéphanie \textsc{Beauceron}, Nazila \textsc{Mahmoudi}, Bastien \textsc{Voirin}, Brigitte \textsc{Cheynis}, Grégoire \textsc{Pierra} et aussi Jérôme \textsc{Degallaix} du LMA et Benjamin \textsc{Blanco} un stagière de M1 de CMS.

J'ai aussi animé le stand "visualisation de la gravité" de l'association Créativ' Sciences dont l'objectif est d'expliquer les principes de base de la gravité en 2D avec un drap tendu.


%%%%%%%%%%%%%%%%%%%%%%%%%%%%%

\begin{appendix}

%% Travail effectué


\chapter{Résumé du travail effectué}

\section{Bibliographie}

\paragraph{\url{https://tel.archives-ouvertes.fr/tel-03405418}}
\begin{itemize}
	\item Étude du calorilmètre hadronique semi-digital et étude du canal physique 
	$$ \positron \electron \longrightarrow \nnh \ (H \longrightarrow WW \longrightarrow qqqq)$$ 
	au collisionneur circulaire électron positon (CEPC) 
	\item Bing Liu, IP2I
	\item 2020
\end{itemize}

\paragraph{\url{https://tel.archives-ouvertes.fr/tel-02141420}}
\begin{itemize}
	\item Étude des gerbes hadroniques dans un calorimètre à grande granularité et étude du canal $$ \positron \electron \longrightarrow HZ \ (Z \longrightarrow qq) $$ dans les futurs collisionneurs leptoniques
	\item Guillaume Garillot, IPNL
	\item 2019
\end{itemize}

\section{Tutoriels}

\begin{description}
	\item[\texttt{LCIO}] \url{https://github.com/iLCSoft/LCIO}
	\item[\texttt{ILDConfig}] \url{https://github.com/iLCSoft/ILDConfig}
	\item[\texttt{Marlin}] \url{https://github.com/iLCSoft/Marlin}
	\item[\texttt{key4hep}] \url{https://github.com/key4hep/k4MarlinWrapper}
\end{description}

\section{Code initial}

\url{https://github.com/ggarillot/nnhAnalysis/tree/refactor}

\section{Code final}

\url{https://github.com/alexhxia/nnhAnalysis}
\chapter{Organisation du Projet}

\section{Organisation initiale}

% Organisation du projet initial : https://github.com/ggarillot/nnhAnalysis/tree/refactor

\begin{figure}[!ht]
	\centering
	\begin{tikzpicture}

		\tikzstyle{home}=[draw, rectangle, fill=red!50, text=black, rounded corners=3pt]		
		
		\tikzstyle{directory}=[draw, rectangle, fill=gray!30, text=black, rounded corners=3pt]

		\node[home] (A) at (0,0) {\texttt{nnhAnalysis}};
		
		\node[directory] (AM) at (-4,-2.5){\minidstmarker};
		\node[directory] (AP) at (0,-2.5) {\processor};
		\node[directory] (AA) at (4,-2.5) {\analysis};
		
		
		\tikzstyle{linkDir}=[->,dotted,very thick,>=latex]
		
		\draw[linkDir] (A)--(AM);
		\draw[linkDir] (A)--(AP); 
		\draw[linkDir] (A)--(AA);
		
	\end{tikzpicture}
	\caption{
		Organisation des dossiers de mon Projet - \url{https://github.com/ggarillot/nnhAnalysis/tree/refactor}
	}
	\label{orga:init}
\end{figure}

\section{Organisation finale}

% Organisation du projet final : https://github.com/alexhxia/nnhAnalysis

\begin{figure}[h!]
	\center
	\begin{tikzpicture}
	
		\tikzstyle{home}=[draw, rectangle, fill=red!50, text=black, rounded corners=3pt]		
	
		\tikzstyle{directory}=[draw, rectangle, fill=gray!30, text=black, rounded corners=3pt]

		\node[directory] (A) at (0,0) {\texttt{nnhAnalysis}};
		
		\node[directory] (S) at (-4.5,-1.5) {\texttt{nnhScript}};
		\node[directory] (P) at (-1.5,-1.5) {\texttt{nnhProgram}};
		\node[directory] (R) at ( 1.5,-1.5) {\texttt{nnhResult}};
		\node[directory] (T) at ( 4.5,-1.5) {\texttt{nnhTest}};
		
		\node[home] (O) at (-5.5,-3) {\texttt{original}};
		\node[home] (I) at (-1.5,-3) {\texttt{ilcsoft}};
		\node[home] (F) at ( 3.5,-3) {\texttt{fcc}};
		
		\node[directory] (OP) at (-7,-4.5) {\texttt{processor}};
		\node[directory] (OA) at (-5,-4.5) {\texttt{analysis}};
		
		\node[directory] (IP) at (-2.5,-4.5) {\texttt{processor}};
		\node[directory] (IA) at (-0.5,-4.5) {\texttt{analysis}};
		
		\node[directory] (FC) at (1.5,-4.5) {\texttt{convert}};
		\node[directory] (FP) at (3.5,-4.5) {\texttt{processor}};
		\node[directory] (FA) at (5.5,-4.5) {\texttt{analysis}};
		
		
%		\tikzstyle{files}=[rectangle, draw, fill=red!40, text=black, below right]		
%		
%		\node[files] (r) at (-2.2,-2) {
%				\texttt{...}
%		};
%		\node[files, text width=3cm] (s) at (0.7,-2) {
%				\texttt{nnh.sh}
%				\texttt{nnhProcessor.sh}
%				\texttt{nnhAnalysis.sh}
%				\texttt{prepaeBDT.sh}
%				\texttt{launchBDT.sh}
%				\texttt{nnhExport.sh}
%		};
%		\node[files, text width=5cm] (t) at (4.3,-2) {
%				\texttt{testProcessorCompleted.py}
%				\texttt{testProcessorSame.py}
%				\texttt{testAnalysisCompleted.py}
%				\texttt{testAnalysisSame.py}
%		};
		
		
		\tikzstyle{linkDir}=[->,dotted,very thick,>=latex]
		
		\draw[linkDir] (A)--(P);
		\draw[linkDir] (A)--(R); 
		\draw[linkDir] (A)--(S);
		\draw[linkDir] (A)--(T); 
		
		\draw[linkDir] (P)--(O);
		\draw[linkDir] (P)--(I); 
		\draw[linkDir] (P)--(F);
		
		\draw[linkDir] (O)--(OP);
		\draw[linkDir] (O)--(OA); 
		\draw[linkDir] (I)--(IP);
		\draw[linkDir] (I)--(IA); 
		\draw[linkDir] (F)--(FP);
		\draw[linkDir] (F)--(FA); 
		\draw[linkDir] (F)--(FC);
		
	\end{tikzpicture}
	\caption{
		Organisation des dossiers de mon Projet - \url{https://github.com/alexhxia/nnhAnalysis}
	}
	\label{orga:end}
\end{figure}

\section{Le dossier \texttt{NNH\_HOME}}

Pour s'exécuter, le projet a besoin de la variable d'environnement \texttt{NNH\_HOME} qui est le chemin du programme que vous souhaitez exécuter,  mis en avant en rouge dans les Figure~\ref{orga:begin} et Figure~\ref{orga:end}.

Donc dans le projet initial, il s'agissait de \texttt{NNH\_HOME=$\backslash$nnhAnalysis} et dans le nouveau projet :
\begin{itemize}
	\item \texttt{NNH\_HOME = $\backslash$nnhAnalysis$\backslash$nnhProgram$\backslash$original}
	\item \texttt{NNH\_HOME = $\backslash$nnhAnalysis$\backslash$nnhProgram$\backslash$ilcsoft}
	\item \texttt{NNH\_HOME = $\backslash$nnhAnalysis$\backslash$nnhProgram$\backslash$fcc}
\end{itemize}
%
\chapter{Fichiers \ROOT de sorties du programme \processor}

On a vu précédemment que \processor converti une série de fichier \SLCIO en fichiers \ROOT. 

Il considère deux canaux de désintégration du Higgs :
\begin{align}
	& h \longrightarrow b \bbar \label{eq:bbar}\\
	& h \longrightarrow W \Wstar \longrightarrow qqqq \label{eq:WWstar}
\end{align}

La première équation (\ref{eq:bbar}) sera mesuré par le détecteur sous la forme de 2 jets reconstruit et identifié comme 2 quark \particle{b}.

Alors que la seconde équation (\ref{eq:WWstar}) sera reconstruit comme 4 jets, 2 par boson \particle{W}\footnote{\particle{W} sur couche de masse, \particle{Wstar} hors couche de masse}.



\end{appendix}

%%%%%%%%%%%%%%%%%%%%%%%%%%%%% Figure %%%%%%%%%%%%%%%%%%%%%%%%%%%%%

%\listoffigures

%%%%%%%%%%%%%%%%%%%%%%%%%%%%% BIBLIO %%%%%%%%%%%%%%%%%%%%%%%%%%%%%

%\bibliographystyle{plain} %{Nabbrv}
%\bibliography{../Bibliographies/biblio}

%\printbibheading


%\nocite{*} % Afficher toute la biblio

\bibliographystyle{plain} %{Nabbrv}
\bibliography{biblio}
%\printbibliography[keyword = {sdhcal}, title = {SDHCAL}]
%\printbibliography[keyword = {ilcsoft}, title = {iLCSoft}]
%\printbibliography[keyword = {fcc}, title = {FCC}]
%\printbibliography[keyword = {particles}, title = {Physique des Particules}]


\end{document}
