\documentclass[10pt,a4paper]{report}

\usepackage[utf8]{inputenc}
\usepackage[french]{babel}
\usepackage[T1]{fontenc}
\usepackage{amsmath}
\usepackage{amsfonts}
\usepackage{amssymb}
\usepackage{graphicx}
\usepackage{lmodern}
\usepackage{hyperref}
\usepackage{url}
\usepackage{xspace}
\usepackage{tikz}

\usepackage[left=2cm,right=2cm,top=2cm,bottom=2cm]{geometry}

% biblio
%\usepackage{biblatex}
%\addbibresource{biblio.bib}

%%%%%%%%%%%%%%%%%%%%%%%%%%%%% Page de garde %%%%%%%%%%%%%%%%%%%%%%%%%%%%%
\author{Alexia \textsc{HOCINE}}

\title{Comparaison des prédictions des suites logicielles\\
	de ILC (iLCSoft) et de FCC (key4HEP) \\
	sur un signal $ e^{+} e^{-} \longrightarrow Z H $ 
}
%Stage M2 Physique, parcours SUBA\\Université de Claude Bernard Lyon 1
\date{Juillet 2022}

%%%%%%%%%%%%%%%%%%%%%%%%%%%%% Raccourçi %%%%%%%%%%%%%%%%%%%%%%%%%%%%%

% raccourci français
\newcommand{\cad}{c'est-à-dire\xspace}
\newcommand{\MS}{Modèle Standard\xspace}


% nom informatique
\newcommand{\ROOT}{\texttt{ROOT}\xspace}
\newcommand{\SLCIO}{\texttt{SLCIO}\xspace}
\newcommand{\iLCSoft}{\texttt{iLCSoft}\xspace}
\newcommand{\LCIO}{\texttt{LCIO}\xspace}
\newcommand{\Marlin}{\texttt{Marlin}\xspace}
%\newcommand{\Key4HEP}{\texttt{Key4HEP}\xspace}
\newcommand{\Gaudi}{\texttt{Gaudi}\xspace}


\newcommand{\nnhAnalysis}{\texttt{nnhAnalysis}\xspace}

\newcommand{\original}{\texttt{original}\xspace}
\newcommand{\ilcsoft}{\texttt{ilcsoft}\xspace}
\newcommand{\fcc}{\texttt{fcc}\xspace}

\newcommand{\minidstmarker}{\texttt{miniDSTMaker}\xspace}
\newcommand{\convert}{\texttt{convert}\xspace}
\newcommand{\processor}{\texttt{processor}\xspace}
\newcommand{\analysis}{\texttt{analysis}\xspace}


% nom de particules
\newcommand{\bbar}{\overline{b}}

%%%%%%%%%%%%%%%%%%%%%%%%%%%%% corps du document %%%%%%%%%%%%%%%%%%%%%%%%%%%%%

\begin{document}

\maketitle

%%%%%%%%%%%%%%%%%%%%%%%%%%%%%

\chapter*{Préambule}

\section*{Remerciements}

Je souhaite d'abord remercier Gérald \textsc{Grenier} pour m'avoir donner la chance de montrer ce que je peux faire. Et aussi pour son encadrement, son accompagnement et son temps.

Je souhaite plus largement remercier mon équipe, Gérald \textsc{Grenier}, Imad \textsc{Laktineh} et Clément \textsc{Devanne}, pour l'atmosphère positive, détendue et stimulante.

Et plus largement, les employés de l'IP2I pour leur gentillesse et leur accueil.

\section*{Participation à la \textit{Geek and Japan Touch}}

Au cours de mon stage, j'ai participé à l'atelier tenu par l'IP2I à la \textit{Geek and Japon Touch}, organisé par Stéphanie \textsc{Beauceron} (IP2I, CNRS, CMS).

Durant ce week-end, avec 2 autres chercheuses (non physiciennes), Florence \textsc{Boyer} et Liliane \textsc{De Araujo}, on a tenu un débat sur le film \textit{Don't look up : Déni Cosmique} de Adam \textsc{McKay}, sur la crédibilité du discours scientifiques.

Ensuite sur le stand, j'ai pu expliquer les bases scientifiques et des recherches menées par le CNRS et CMS, ainsi que Virgo au près du grand publique. 

De plus, j'ai aussi animé le stand de l'association ?? dont l'objectif est d'expliquer les principes de base de la gravité en 2D avec un drap tenu.

%%%%%%%%%%%%%%%%%%%%%%%%%%%%%

\tableofcontents

%%%%%%%%%%%%%%%%%%%%%%%%%%%%%

\chapter{Introduction}

\section{Physique des collisionneurs}

Le principe des collisionneurs est simple, on accélère des particules à des 
énergies cinétiques suffisantes pour provoquer des collisions inélastiques, et ainsi comprendre les interactions fondamentales et les constituants élémentaires de la physique.\\

On distingue 2 familles de collisionneurs en fonction des particules qui sont utilisés.

\subsection{Collisionneurs hadroniques}

Les collisionneurs hadroniques utilisent des hadrons, qui sont des 
particules complexes composées de 3 quarks et de gluons\footnote{Gluon : boson médiateursde l'interaction forte qui maintiennent les quarks ensembles.}. 
En pratique au LHC (Large Hadron Collider) du CERN, on utilise des protons, 2 quarks up et un quark down. 

Comme il s'agit de particules composites, ce sont pas le protons qui collisionnent directement mais ces contituants, appelés partons. Chacun porte une fraction indéterminée de l'énergie du proton. 
Ce qui permet d'avoir des énergies de collisions inconnues en amont.
C'est pourquoi, ils sont utiles pour la découverte de nouvelles particules de masse inconnue, puisqu'il permettent de balayer tout le spectre de masse sous la gamme d'énergie du collisionneur (au LHC < 14 TeV)\footnote{D'où l'intérêt de nouveaux collisionneurs à des énergies plus élevées et donc des masses de particules produites plus lourdes.}.

\subsection{Collisionneurs leptoniques}

En revanche, les collisionneurs leptoniques utilisent des leptons, qui sont des particules élémentaires. Comme le LEP (Large Electron-Positron), le prédécesseur du LHC, qui collisionnait des électrons et des positrons \cite{cern:lep}.

Cette fois-ci, chaque lepton qui collisionne, possède une énergie complète. Donc on connait parfaitement leur énergie, puisque on peut leurs impulser une énergie choisie et ainsi augmenter la statistique pour un niveau d'énergie précis.
Ces collisionneurs sont donc utilisés pour la recherche de précision.\\


Les prochaines générations de collisionneurs, comme ILC, CEPC, CLIC et FCC, ce sont des collisionneurs leptoniques. 
Leur objectif est de préciser les données du LHC, notamment sur le boson de Higgs découvert en 2012 par le LHC, qui était la pièce manquante du modèle standard des particules, car il permet aux particules d'acquérir une masse.

\section{Physique du boson de Higgs}

\subsection{Production du boson de Higgs}

Concrètement, on ne mesure pas directement le boson de Higgs mais les particules qu'il produit sous la forme de jets.
Ainsi on veut améliorer la résolutions en énergie de ces jets que l'on détecte \cite{liu:tel-03405418}.

\subsection{Détecteur}

En physique des particules, on utilise des détecteurs appelés calorimètres pour mesurer l'énergie des particules. 
Cette énergie va être déposer par ionisation avec le matériau le long de la trajectoire des particules qui le traverse. 
Il faut donc des algorithmes de reconstruction pour déduire les énergies, les types de particules et les trajectoires.

Pour cela, on utilise des calorimètres à grande granularité qui permet une très bonne performance des Algorithmes de Flux de Particules (PFA) \cite{liu:tel-03405418}. \\

C'est dans ce cadre que la collaboration internationale CALICE, à développer le premier prototype de la famille de calorimètre granulaire SDHCAL, pour Semi-Digital Hadronic CALorimeter, qui a été développer en grande partie à l'IP2I dans l'équipe CMS, auquel j'appartiens pour ce stage.

\subsection{Collisions}

Au cours, de ce stage, je me concentrerais sur les collisions de type \texttt{nnh} pour neutrino-neutrino-higgs.

D'où voici les diagrammes de Feynman :

\begin{figure}
	\centering
	\begin{tikzpicture}
	
	\end{tikzpicture}
	\label{feynmann}
\end{figure}

\section{Présentation \& Objectif du Stage}

Pour ce stage, j'ai récupéré les codes de Guillaume Garillot, qui les a développé en 2021 au cours de son post-doctorat à l'IP2I. 
Ils sont en libre accès à l'adresse \url{https://github.com/ggarillot/nnhAnalysis/tree/refactor}.\\

Ce programme \nnhAnalysis permet l'étude de fichiers \SLCIO pour la collision :
\begin{equation}
	e^{+} e^{-} \longrightarrow \nu \nu h 
\end{equation}

Et l'analyse des canaux de désintégration :

\begin{align}
	h &\longrightarrow WW \longrightarrow qqqq \\
	h &\longrightarrow b \bbar 
\end{align}

Pour cela, il a utilisé les suites logiciels de \iLCSoft, \url{https://github.com/iLCSoft} (plus précisément \LCIO et \Marlin), qui sont les anciennes suites logicielles.
Mais les nouveaux projets de collisionneurs changent de suites logicielles et passent à \texttt{Key4HEP} et \Gaudi.\\

Mon objectif est double. Dans un premier temps, comprendre et optimiser les codes existants, \cad les programmes \LCIO, \Marlin et ceux de \nnhAnalysis. Puis, je vais devoir transformer \nnhAnalysis pour qu'il puisque correspondre aux nouvelles normes \texttt{Key4HEP} et \Gaudi.

%%%%%%%%%%%%%%%%%%%%%%%%%%%%%

\chapter{\texttt{ILC}}

\section{Présentation du Projet ILC}

Le projet ILC (International Linear Collider) est un collisionneur linéaire, électron-positron, de 31 km conçu pour atteindre une énergie de centre de masse de 500 GeV \cite{cern:ilc}. \\

L'objectif de l'ILC est de produire beaucoup de boson de Higgs afin de rechercher de la nouvelle physique, par de nouveaux écarts avec le \MS et il va surtout essayer de découvrir s'il y en a d'autre génération du boson de Higgs. \\

Ce projet est toujours en attente pour commencer sa construction, probablement dans les montage du Nord du Japon. Et le détecteur SDHCAL est en course pour y être installer. C'est pourquoi, l'IP2I développe dans programme d'analyse en parallèle du détecteur.

\begin{figure}[h!]
	\center
	\includegraphics[width=\textwidth]{../img/ilc.jpg} 
	\caption{Schéma ILC\cite{cern:ilc}}
	\label{ilc:schema}
\end{figure}

\section{Projet numérique : \original}

Pour ce stage j'ai récupéré les codes de Guillaume Garillot, qui les a développé en 2021 au cours de son post-doctorat à l'IP2I. 

\subsection{Programme : \minidstmarker}

Initialement, 


on m'a mis à disposition des fichiers \SLCIO, qui sera le format de fichier du détecteur.
Chaque fichier est rangés dans un des 66 dossiers (Figure~\ref{listeProcessus}), qui correspond au code du type de processus.

\begin{figure}[h!]
	\includegraphics[width=\textwidth]{../img/listeProcessus.png} 
	\caption{Les noms des dossiers qui correspondent aux numéros de processus}
	\label{listeProcessus}
\end{figure}

%\paragraph{Numéro des processus ???}

\subsection{Programme : \processor}

\subsubsection{Méthodes}

On cherche à convertir ces fichiers \SLCIO en arbre \ROOT par processus.

\subsubsection{Résultats}

%\paragraph{Des fichiers \ROOT :}
Chaque dossier de fichier de donnée \SLCIO produira un fichier \ROOT en sortie, \cad que l'on obtiendra un arbre \ROOT par processus.


\subsubsection{Interprétation}

\subsection{Programme \analysis}

\subsubsection{Données}

On récupère les fichiers \ROOT du programme \processor précédent. 

$ hadd $ qui va créer le fichier DATA.root

\subsubsection{Méthodes}

\paragraph{\texttt{BDT}}

Entrainement

\paragraph{L'analyse}

\section{Projet numérique \ilcsoft}

\subsection{Données}
Initialement, on m'a mis à disposition des fichiers \SLCIO, qui sera le format de fichier du détecteur.
Chaque fichier est rangés dans un des 66 dossiers (Figure~\ref{listeProcessus}), qui correspond au code du type de processus.

\begin{figure}[h!]
	\includegraphics[width=\textwidth]{../img/listeProcessus.png} 
	\caption{Les noms des dossiers qui correspondent aux numéros de processus}
	\label{listeProcessus}
\end{figure}

%\paragraph{Numéro des processus ???}

\subsection{Programme : \processor}

\subsubsection{Méthodes}

On cherche à convertir ces fichiers \SLCIO en arbre \ROOT par processus.

\subsubsection{Résultats}

%\paragraph{Des fichiers \ROOT :}
Chaque dossier de fichier de donnée \SLCIO produira un fichier \ROOT en sortie, \cad que l'on obtiendra un arbre \ROOT par processus.


\subsubsection{Interprétation}

\subsection{Programme \analysis}

\subsubsection{Données}

On récupère les fichiers \ROOT du programme \processor précédent. 

$ hadd $ qui va créer le fichier DATA.root

\subsubsection{Méthodes}

\paragraph{\texttt{BDT}}

Entrainement

\paragraph{L'analyse}


\subsubsection{Résultats}

\paragraph{Vérification des résultats}
Comparaison entre les différents séries d'analyse, basée sur les même fichiers \ROOT, mais un autre entraînement de BDT.

\subsubsection{Interprétation}

%%%%%%%%%%%%%%%%%%%%%%%%%%%%%

\chapter{\texttt{FCC}}

\section{Projet FCC}

\subsection{Présentation}

Le FCC (Futur Collisionneur Circulaire) est le projet du CERN pour remplacer 
leur collisionneur actuelle, le LHC (Large Hadronic Collider). 
Dont la fin de l'exploitation est prévu en 2040 \cite{cern:fcc}

Pour le FCC, on prévoit un anneau de 100 km, contre 27 km pour le LEP et le LHC 
(comme montrer Figure~\ref{fcc:img}).
Ce qui devrait nous permettra d'atteindre une énergie de 100 TeV contre 13 TeV
actuellement pour le LHC.

L'objectif est de rechercher d'une nouvelle physique, en mettant au jour de 
déviation avec le modèle standard.


\begin{figure}[h!]
    \centering
    \includegraphics[width=\textwidth]{../img/FCC.jpg}
    \caption{https://cds.cern.ch/images/OPEN-PHO-ACCEL-2019-001-2}
    \label{fcc:img}
\end{figure}

\section{Développement Numérique}

Mon objectif dans ce stage est de transformer les codes développés par Guillaume 
\textsc{Garillot} lors de son post-doctorat pour le projet ILC pour ce projet 
qui n'utilise pas les mêmes suites logiciels.

\texttt{Gaudi}

\texttt{EDM4hep}

\section{Travail de Stage}

\section{Comparaison avec \texttt{iLCSoft}}

%%%%%%%%%%%%%%%%%%%%%%%%%%%%%

\chapter{Outils Numériques}

\section{\texttt{nnhScript}}
\url{https://github.com/alexhxia/nnhAnalysis/tree/main/nnhScript}

\section{\texttt{nnhTest}}

Pour tester les programmes générer avec \texttt{nnhProgram}, j'ai développé 4 programmes en \texttt{python} : 

\begin{figure}[h!]
	\center
	\begin{tabular}{| c | c | c |}
		\hline
			\texttt{•} & \texttt{Processus} & \texttt{Analysis} \\
		\hline
			\texttt{Completed} & \texttt{testProcessorCompleted.py} & \texttt{testAnalysisCompleted.py} \\
		\hline
			\texttt{Same} & \texttt{testProcessorSame.py} & \texttt{testAnalysisSame.py} \\
		\hline
	\end{tabular}
	\caption{Tableau récapitulatif des fonctions de tests}
\end{figure}

\subsection{Programmes \texttt{testXxCompleted.py}}

L'objectif de ce type de programme est de tester si tous les fichiers ont été généré.

\subsubsection{Programmes \texttt{testProcessorCompleted.py}}

Le processus est complet si tous les dossiers de 

\subsubsection{Programmes \texttt{testAnalysisCompleted.py}}


\subsection{Programmes \texttt{testXxSame.py}}

\subsubsection{Programmes \texttt{testProcessorSame.py}}

\subsubsection{Programmes \texttt{testAnalysisSame.py}}



\url{https://github.com/alexhxia/nnhAnalysis/tree/main/nnhTest}

%%%%%%%%%%%%%%%%%%%%%%%%%%%%%

\chapter{Exemples de Résultats Physiques}

%%%%%%%%%%%%%%%%%%%%%%%%%%%%%
\chapter{Conclusion}

\section{Résumer du travail effectué}

Mon stage est principalement du développement numérique, et l'ensemble des codes à conserver par l'IP2I sont sur mon GitHub publique : 

\url{https://github.com/alexhxia/nnhAnalysis}

\paragraph{Email Gérald Grenier :}

\subparagraph{Un tutorial de \texttt{ilcsoft} :} 
\url{https://agenda.linearcollider.org/event/9272/}

\subparagraph{Initialisation \texttt{ilcsoft} :}
%source /cvmfs/ilc.desy.de/sw/x86_64_gcc82_centos7/v02-02-03/init_ilcsoft.sh

\subparagraph{La documentation et le packet \texttt{git} du format de données \texttt{LCIO} et de la librairie \texttt{Marlin}}
\begin{itemize}
	\item \url{https://github.com/iLCSoft/LCIO} \cite{Gaede:2003ip}
	\item \url{https://github.com/iLCSoft/Marlin}
\end{itemize}


\subparagraph{Pour la deuxième partie du stage :}
\begin{itemize}

	\item le software en développement : 
			\url{https://github.com/key4hep}
			
	\item et plus particulièrement l'adaptateur \texttt{ilcsoft} vers \texttt{key4hep} : 
			\url{https://github.com/key4hep/k4MarlinWrapper}
			
\end{itemize}  

%%%%%%%%%%%%%%%%%%%%%%%%%%%%%

\begin{appendix}

\chapter{Organisation du Projet}

\section{Organisation initiale}

\begin{figure}[h!]
	\center
	\begin{tikzpicture}

		\tikzstyle{home}=[draw, rectangle, fill=red!50, text=black, rounded corners=3pt]		
		
		\tikzstyle{directory}=[draw, rectangle, fill=gray!30, text=black, rounded corners=3pt]

		\node[home] (A) at (0,0) {\texttt{nnhAnalysis}};
		
		\node[directory] (AM) at (-4,-2.5){\minidstmarker};
		\node[directory] (AP) at (0,-2.5) {\processor};
		\node[directory] (AA) at (4,-2.5) {\analysis};
		
		
		\tikzstyle{linkDir}=[->,dotted,very thick,>=latex]
		
		\draw[linkDir] (A)--(AM);
		\draw[linkDir] (A)--(AP); 
		\draw[linkDir] (A)--(AA);
		
	\end{tikzpicture}
	\caption{
		Organisation des dossiers de mon Projet - \url{https://github.com/ggarillot/nnhAnalysis/tree/refactor}
	}
	\label{orga:begin}
\end{figure}

\section{Organisation finale}

\begin{figure}[h!]
	\center
	\begin{tikzpicture}
	
		\tikzstyle{home}=[draw, rectangle, fill=red!50, text=black, rounded corners=3pt]		
	
		\tikzstyle{directory}=[draw, rectangle, fill=gray!30, text=black, rounded corners=3pt]

		\node[directory] (A) at (0,0) {\texttt{nnhAnalysis}};
		
		\node[directory] (S) at (-4.5,-1.5) {\texttt{nnhScript}};
		\node[directory] (P) at (-1.5,-1.5) {\texttt{nnhProgram}};
		\node[directory] (R) at ( 1.5,-1.5) {\texttt{nnhResult}};
		\node[directory] (T) at ( 4.5,-1.5) {\texttt{nnhTest}};
		
		\node[home] (O) at (-5.5,-3) {\texttt{original}};
		\node[home] (I) at (-1.5,-3) {\texttt{ilcsoft}};
		\node[home] (F) at ( 3.5,-3) {\texttt{fcc}};
		
		\node[directory] (OP) at (-7,-4.5) {\texttt{processor}};
		\node[directory] (OA) at (-5,-4.5) {\texttt{analysis}};
		
		\node[directory] (IP) at (-2.5,-4.5) {\texttt{processor}};
		\node[directory] (IA) at (-0.5,-4.5) {\texttt{analysis}};
		
		\node[directory] (FC) at (1.5,-4.5) {\texttt{convert}};
		\node[directory] (FP) at (3.5,-4.5) {\texttt{processor}};
		\node[directory] (FA) at (5.5,-4.5) {\texttt{analysis}};
		
		
%		\tikzstyle{files}=[rectangle, draw, fill=red!40, text=black, below right]		
%		
%		\node[files] (r) at (-2.2,-2) {
%				\texttt{...}
%		};
%		\node[files, text width=3cm] (s) at (0.7,-2) {
%				\texttt{nnh.sh}
%				\texttt{nnhProcessor.sh}
%				\texttt{nnhAnalysis.sh}
%				\texttt{prepaeBDT.sh}
%				\texttt{launchBDT.sh}
%				\texttt{nnhExport.sh}
%		};
%		\node[files, text width=5cm] (t) at (4.3,-2) {
%				\texttt{testProcessorCompleted.py}
%				\texttt{testProcessorSame.py}
%				\texttt{testAnalysisCompleted.py}
%				\texttt{testAnalysisSame.py}
%		};
		
		
		\tikzstyle{linkDir}=[->,dotted,very thick,>=latex]
		
		\draw[linkDir] (A)--(P);
		\draw[linkDir] (A)--(R); 
		\draw[linkDir] (A)--(S);
		\draw[linkDir] (A)--(T); 
		
		\draw[linkDir] (P)--(O);
		\draw[linkDir] (P)--(I); 
		\draw[linkDir] (P)--(F);
		
		\draw[linkDir] (O)--(OP);
		\draw[linkDir] (O)--(OA); 
		\draw[linkDir] (I)--(IP);
		\draw[linkDir] (I)--(IA); 
		\draw[linkDir] (F)--(FP);
		\draw[linkDir] (F)--(FA); 
		\draw[linkDir] (F)--(FC);
		
	\end{tikzpicture}
	\caption{
		Organisation des dossiers de mon Projet - \url{https://github.com/alexhxia/nnhAnalysis}
	}
	\label{orga:end}
\end{figure}

\section{Le dossier \texttt{NNH\_HOME}}

Pour s'exécuter, le projet a besoin de la variable d'environnement \texttt{NNH\_HOME} qui est le chemin du programme que vous souhaitez exécuter,  mis en avant en rouge dans les Figure~\ref{orga:begin} et Figure~\ref{orga:end}.

Donc dans le projet initial, il s'agissait de \texttt{NNH\_HOME=$\backslash$nnhAnalysis} et dans le nouveau projet :
\begin{itemize}
	\item \texttt{NNH\_HOME = $\backslash$nnhAnalysis$\backslash$nnhProgram$\backslash$original}
	\item \texttt{NNH\_HOME = $\backslash$nnhAnalysis$\backslash$nnhProgram$\backslash$ilcsoft}
	\item \texttt{NNH\_HOME = $\backslash$nnhAnalysis$\backslash$nnhProgram$\backslash$fcc}
\end{itemize}


\end{appendix}

%%%%%%%%%%%%%%%%%%%%%%%%%%%%% Figure %%%%%%%%%%%%%%%%%%%%%%%%%%%%%

\listoffigures

%%%%%%%%%%%%%%%%%%%%%%%%%%%%% BIBLIO %%%%%%%%%%%%%%%%%%%%%%%%%%%%%

%\bibliographystyle{plain} %{Nabbrv}
%\bibliography{../Bibliographies/biblio}

%\printbibheading


\nocite{*} % Afficher toute la biblio

\bibliographystyle{plain} %{Nabbrv}
\bibliography{biblio}
%\printbibliography[keyword = {sdhcal}, title = {SDHCAL}]
%\printbibliography[keyword = {ilcsoft}, title = {iLCSoft}]
%\printbibliography[keyword = {fcc}, title = {FCC}]
%\printbibliography[keyword = {particles}, title = {Physique des Particules}]


\end{document}
