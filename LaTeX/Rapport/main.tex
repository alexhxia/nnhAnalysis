\documentclass[10pt,a4paper]{report}

\usepackage[utf8]{inputenc}
\usepackage[french]{babel}
\usepackage[T1]{fontenc}
\usepackage{amsmath}
\usepackage{amsfonts}
\usepackage{amssymb}
\usepackage{graphicx}
\usepackage{lmodern}
\usepackage{hyperref}
\usepackage{url}
\usepackage{xspace}

\usepackage[left=2cm,right=2cm,top=2cm,bottom=2cm]{geometry}

% biblio
%\usepackage{biblatex}
%\addbibresource{biblio.bib}

% Page de garde
\author{Alexia \textsc{HOCINE}}

\title{Comparaison des prédictions des suites logicielles\\
	de ILC (iLCSoft) et de FCC (key4HEP) \\
	sur un signal $ e^{+} e^{-} \longrightarrow Z H $ 
}
%Stage M2 Physique, parcours SUBA\\Université de Claude Bernard Lyon 1
\date{Juillet 2022}

% raccourci français
\newcommand{\cad}{c'est-à-dire\xspace}

% nom informatique
\newcommand{\ROOT}{\texttt{ROOT}\xspace}
\newcommand{\SLCIO}{\texttt{SLCIO}\xspace}
\newcommand{\processor}{\texttt{processor}\xspace}
\newcommand{\analysis}{\texttt{analysis}\xspace}
\newcommand{\nnhAnalysis}{\texttt{nnhAnalysis}\xspace}

% nom de particules
\newcommand{\bbar}{\overline{b}}

% corps du document
\begin{document}

\maketitle

% Préambule

\chapter*{Préambule}

\section*{Remerciements}

Je souhaite d'abord remercier Gérald \textsc{Grenier} pour m'avoir donner la chance de montrer ce que je peux faire. Et aussi pour son encadrement, son accompagnement et son temps.

Je souhaite plus largement remercier mon équipe, Gérald \textsc{Grenier}, Imad \textsc{Laktineh} et Clément \textsc{Devanne}, pour l'atmosphère positive, détendue et stimulante.

Et plus largement, les employés de l'IP2I pour leur gentillesse et leur accueil.

\section*{Participation à la \textit{Geek and Japan Touch}}

Au cours de mon stage, j'ai participé à l'atelier tenu par l'IP2I à la \textit{Geek and Japon Touch}, organisé par Stéphanie \textsc{Beauceron} (IP2I, CNRS, CMS).

Durant ce week-end, avec 2 autres chercheuses (non physiciennes), Florence \textsc{Boyer} et Liliane \textsc{De Araujo}, on a tenu un débat sur le film \textit{Don't look up : Déni Cosmique} de Adam \textsc{McKay}, sur la crédibilité du discours scientifiques.

Ensuite sur le stand, j'ai pu expliquer les bases scientifiques et des recherches menées par le CNRS, CMS, et Virgo au près du grand publique. 

De plus, j'ai aussi animé le stand de l'association ?? dont l'objectif est d'expliquer les principes de base de la gravité en 2D avec un drap tenu.

\tableofcontents

\chapter{Introduction}

\section{Objectifs physiques}

\subsection{Collisions}
Au cours, de ce stage, je me concentrerais sur les collisions de type \texttt{nnh} pour neutrino-neutrino-higgs

\section{SDHCAL (Semi-Digital Hadronic CALorimeter)}

tests en Septembre

\section{iLCSoft}

\section{FCC}

\section{Présentation \& Objectif du Stage}



\chapter{\texttt{ilcsoft}}

\section{Projet \nnhAnalysis}

\section{Programme \processor}

\subsubsection{Données}
Initialement, on m'a mis à disposition des fichiers \SLCIO rangés par processus dans 66 dossiers (Figure~\ref{listeProcessus}).

\begin{figure}[h!]
	%\includegraphics[width=\textwidth]{../img/listeProcessus.png} 
	\caption{Les noms des dossiers qui correspondent aux numéros de processus}
	\label{listeProcessus}
	
\end{figure}

\paragraph{Numéro des processus ???}

%\paragraph{Résultats mis à disposition ???}

\subsubsection{Méthodes}

On cherche à convertir ces fichiers \SLCIO en arbre \ROOT par processus.

\subsubsection{Résultats}

%\paragraph{Des fichiers \ROOT :}
Chaque dossier de fichier de donnée \SLCIO produira un fichier \ROOT en sortie, \cad que l'on obtiendra un arbre \ROOT par processus.


\subsubsection{Interprétation}

\section{Programme \analysis}

\subsubsection{Données}

On récupère les fichiers \ROOT du programme \processor précédent. 

$ hadd $ qui va créer le fichier DATA.root

\subsubsection{Méthodes}

\paragraph{\texttt{BDT}}

Entrainement

\paragraph{L'analyse}



\subsubsection{Résultats}

\paragraph{Vérification des résultats}
Comparaison entre les différents séries d'analyse, basée sur les même fichiers \ROOT, mais un autre entraînement de BDT.

\subsubsection{Interprétation}

\chapter{Programme \texttt{FCC}}

\section{Présentation du Projet FCC}

Le FCC (Futur Collisionneur Circulaire) est le projet du CERN pour remplacer 
leur collisionneur actuelle, le LHC (Large Hadronic Collider). 
Dont la fin de l'exploitation est prévu en 2040 \cite{cern:fcc}.
On prévoit un anneau de 100 km, contre 27 km pour le LEP et le LHC 
(comme montrer Figure~\ref{fcc:img}).
Ce qui devrait nous permettra d'atteindre une énergie de 100 $\TeV$ contre 13 $\TeV$
actuellement pour le LHC.

L'objectif est la rechercher d'une nouvelle physique par la mise en évidence de déviation avec le modèle standard. Et plus particulièrement, en augmentant la statistique sur le boson de Higgs, découvert avec le collisionneur actuelle, afin de mieux comprendre sa physique.

\begin{figure}[!ht]
    \centering
    \includegraphics[width=\textwidth]{../img/FCC.jpg}
    \caption{https://cds.cern.ch/images/OPEN-PHO-ACCEL-2019-001-2}
    \label{fcc:img}
\end{figure}

Une fois encore, le détecteur SDHCAL candidate pour ce projet et pourrait être y être installer.

\section{Développement Numérique}

Mon objectif est d'adapter les codes développés précédemment par Guillaume 
\textsc{Garillot} du projet ILC au projet FCC qui n'utilise pas les mêmes suites logiciels.\\


\subsection{\convert}

Dans un premier temps, je dois convertir les fichiers \SLCIO en fichiers \ROOT en utilisant le programme libre \texttt{k4MarlinWrapper} du projet de \texttt{key4hep}\footnote{\url{https://github.com/key4hep/k4MarlinWrapper}}. 
Il s'agit d'une collaboration entre chercheurs du CERN dont l'objectif est de développer des outils pour le projet FCC, y compris des utils de conversion entre \LCIO (\iLCSoft) et \EDMhep (\FCC).%, en utilisant les algorithmes \texttt{Gaudi}
\\

Cette partie a été laborieux, car le programme \texttt{k4MarlinWrapper} est en cours de développement. 
J'ai du faire remonter de très nombreux problèmes en parallèle de mon travail. 
En effet, à chaque mise à jour, le programme ne fonctionnait plus, ce qui m'a fait perdre beaucoup de temps en essayant de comprendre le problème, le signaler et le corriger quand je le comprenais, ou patienter pour que quelqu'un d'autre le corrige. 

\subsection{\processor}

Comme les fichiers d'entrées sont à présent des fichiers \ROOT, il faut adapter cette partie pour que les fichiers de sortie soit les mêmes ou au moins équivalent.

\subsection{\analysis}

Cette partie reste inchangé par rapport à \ilcsoft.

\chapter{Outils Numériques}

\section{\texttt{nnhScript}}
\url{https://github.com/alexhxia/nnhAnalysis/tree/main/nnhScript}

\section{\texttt{nnhTest}}

Pour tester les programmes générer avec \texttt{nnhProgram}, j'ai développé 4 programmes en \texttt{python} : 

\begin{figure}[h!]
	\center
	\begin{tabular}{| c | c | c |}
		\hline
			\texttt{•} & \texttt{Processus} & \texttt{Analysis} \\
		\hline
			\texttt{Completed} & \texttt{testProcessorCompleted.py} & \texttt{testAnalysisCompleted.py} \\
		\hline
			\texttt{Same} & \texttt{testProcessorSame.py} & \texttt{testAnalysisSame.py} \\
		\hline
	\end{tabular}
	\caption{Tableau récapitulatif des fonctions de tests}
\end{figure}

\subsection{Programmes \texttt{testXxCompleted.py}}

L'objectif de ce type de programme est de tester si tous les fichiers ont été généré.

\subsubsection{Programmes \texttt{testProcessorCompleted.py}}

Le processus est complet si tous les dossiers de 

\subsubsection{Programmes \texttt{testAnalysisCompleted.py}}


\subsection{Programmes \texttt{testXxSame.py}}

\subsubsection{Programmes \texttt{testProcessorSame.py}}

\subsubsection{Programmes \texttt{testAnalysisSame.py}}



\url{https://github.com/alexhxia/nnhAnalysis/tree/main/nnhTest}

\listoffigures

%%%%% BIBLIO %%%%%

%\bibliographystyle{plain} %{Nabbrv}
%\bibliography{../Bibliographies/biblio}

%\printbibheading

% Afficher toute la biblio
\nocite{*}
\bibliographystyle{plain} %{Nabbrv}
\bibliography{biblio}
%\printbibliography[keyword = {sdhcal}, title = {SDHCAL}]
%\printbibliography[keyword = {ilcsoft}, title = {iLCSoft}]
%\printbibliography[keyword = {fcc}, title = {FCC}]
%\printbibliography[keyword = {particles}, title = {Physique des Particules}]


\end{document}
