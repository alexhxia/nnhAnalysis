% Tableau des codes associés aux résultats des processus

\begin{figure}[h!]
    \centering
    \begin{tabular}{ | l | l | l | }
    		\hline
    		Type de processus & Code des processus \\
        \hline
        2 leptoniques & 500006, 500008 \\
        \hline
        2 hadroniques & 500010, 500012 \\
        \hline
        4 hadroniques & 500062, 500064, 500066, 500068, 500070, 500072 \\
        \hline
        4 semi-leptoniques & 500074, 500076, 500078, 500080, 500082, 500084, 500101, 500102, 500103, 500104, \\ 
        & 500105, 500106, 500107, 500108, 500110, 500112 \\
        \hline
        4 leptoniques & 500086, 500088, 500090, 500092, 500094, 500096, 500098, 500100, 500113, 500114, 500115, \\
        & 500116, 500117, 500118, 500119, 500120, 500122, 500124, 500125, 500126,500127, 500128  \\
        	\hline
        signal & 402007, \color{red}{402173}, \color{blue}{402176}\\
        \hline
        autres higgs & 402001, 402002, 402003, 402004, 402005, 402006, 402008, 402009, 402010, 402011, 402012, \\ 
        & 402013, 402014, 402182, 402185, \color{blue}{402173}, \color{red}{402176}\\
        \hline
    \end{tabular}
    \caption{Signification des codes des processus, \color{red}{si le signal recherché est de type \bb}, \color{blue}{si le signal est \WW}}
    \label{data:code}
\end{figure}

%if (isBB) 
%        CHANNELS_SIGNAL.insert(402173);
%        CHANNELS_OTHERHIGGS.insert(402176);
%else 
%        CHANNELS_SIGNAL.insert(402176);
%        CHANNELS_OTHERHIGGS.insert(402173);