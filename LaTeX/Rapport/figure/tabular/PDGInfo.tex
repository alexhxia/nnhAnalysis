\begin{figure}[!ht]

	\begin{description}
		\item[Constantes] :\\
		\begin{flushleft}
		\begin{description}
			\item[Leptons]
    				\verb|ELECTRON = 11|,
    				\verb|ELECTRON_NEUTRINO = 12|,
    				\verb|MUON = 13|,
    				\verb|MUON_NEUTRINO = 14|,
    				\verb|TAU = 15|,
    				\verb|TAU_NEUTRINO = 16|

    			\item[Bosons]
    				\verb|GLUON = 21|,
    				\verb|PHOTON = 22|,
    				\verb|Z0 = 23|,
    				\verb|W = 24|,
    				\verb|HIGGS = 25|
		\end{description}
		\end{flushleft}
		
		\item[Requêtes]
		
		\item[Tests] :\newline
		\begin{itemize}
	
			\item Test si un entier correspond au PDG d'un neutrino :\\
\verb|bool isNeutrino(int pdg);|

			\item Test si un entier correspond au PDG d'un lepton chargé :\\
\verb|bool isChargedLepton(int pdg);|

			\item Test si un entier correspond au PDG d'un quark :\\
\verb|bool isQuark(int pdg);|

			\item Test si un entier correspond au PDG d'une particule :\\
\verb|bool isAbsPDG(const int pdg, const EVENT::MCParticle* particle);|

			\item Test si 2 entiers correspondent aux mêmes particules au signe près :\\
\verb|bool isSameParticleAbsPDG(const EVENT::MCParticle* p1, const EVENT::MCParticle* p2);|

			\item Test si 2 entiers correspondent à une paire particule-antiparticule :\\
\verb|bool isTwinParticlePDG(const EVENT::MCParticle* p1, const EVENT::MCParticle* p2);|
    
			\item Retourne la génération d'un code PDG d'un lepton chargé :\\
\verb|int getFlavorLeptonAbs(int pdg);|

			\item Compte le nombre de neutrino :\\
\verb|int getNeutrinoNb(const std::vector<int> vecPDG);|

	\end{itemize}
	\end{description}
	
	\caption{Classe \texttt{PDGInfo}}
	\label{tab:PDGInfo}
\end{figure}