\chapter{Introduction}

\section{Physique des collisionneurs}

Le principe des collisionneurs est simple, on accélère des particules à des 
énergies cinétiques suffisantes pour provoquer des collisions inélastiques.

Il en existe de 2 types que l'on distingue par le type de particules que l'on 
utilise.

\subsection{Collisionneurs hadroniques}

Dans les collisionneurs hadroniques on utilise des hadrons, qui sont des 
particules complexes composées de 3 quarks et de gluons (bosons médiateurs 
de l'interaction forte qui maintiennent les quarks ensembles). 
En pratique au LHC (Large Hadron Collider) du CERN, on utilise des protons, 
2 quarks up et un quark down. 

Comme il s'agit de particules composites, ce sont pas le protons qui colisionnent 
directement mais ces contituants, appelés partons. Chacuns porte une fraction de 
l'énergie du proton. 
Ce qui permet d'avoir des énergies de collisions indéterminées en amont.
C'est pourquoi, ils sont utiles pour la découverte de nouvelles particules de 
masse inconnue, puisqu'il permettent de balayer tout le spectre de masse d'une 
gamme d'énergie.

\subsection{Collisionneurs leptoniques}

En revanche, les collisionneurs leptoniques collisionnent des leptons, qui sont
des particules élémentaires. Donc on connait parfaitement leur énergie et on les 
utilise pour faire de la recherche à un niveau d'énergie choisi.  
Ce sont des collisionneurs de précision.\\

Les prochaines générations de collisionneurs, ILC, CEPC, CLIC et FCC, sont des 
collisionneurs leptoniques. Leur objectif est de préciser les données du LHC, 
notamment sur le boson de Higgs découvert en 2012 par le LHC, qui était la pièce 
manquante du modèle standard des particules, car il permet aux particules 
d'acquérir une masse.

\section{Physique du boson de Higgs}

\subsection{Production du boson de Higgs}

Concrètement, on ne mesure pas directement le boson de Higgs mais les particules 
qu'ils produitent sous la forme de jets.
Ainsi on veut améliorer la résolutions en énergie des jets que l'on détecte \cite{liu:tel-03405418}.


\section{SDHCAL (Semi-Digital Hadronic CALorimeter)}

Pour cela, on utilise ici des calorimètres à grande granularité qui permet une 
très bonne performance des Algorithmes de Flux de Particules (PFA) \cite{liu:tel-03405418}.

Dans le cadre de la collaboration internationale CALICE, le premier prototype de la famille de calorimètres granulaires SDHCAL a été développer au grande partie à l'IP2I.

tests en Septembre

\subsection{Collisions}
Au cours, de ce stage, je me concentrerais sur les collisions de type \texttt{nnh} pour neutrino-neutrino-higgs

%\section{iLCSoft}

%\section{FCC}

\section{Présentation \& Objectif du Stage}

analyse des canaux :
\begin{equation}
	e^{+} e^{-} \longrightarrow \nu \nu h \left(h \longrightarrow WW \longrightarrow qqqq \right)
\end{equation}

\begin{equation}
	e^{+} e^{-} \longrightarrow \nu \nu h \left(h \longrightarrow b \bbar \right)
\end{equation}
