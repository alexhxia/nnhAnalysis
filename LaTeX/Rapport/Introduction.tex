ev Introduction

\chapter{Introduction}

\section{La Physique des Collisionneurs}

Le principe des collisionneurs est simple, on accélère des particules à des 
énergies cinétiques suffisantes pour provoquer des collisions inélastiques, et ainsi comprendre les interactions fondamentales et les constituants élémentaires de la matière.\\

On distingue 2 familles de collisionneurs en fonction des particules qui sont utilisés.

\subsection{Collisionneurs hadroniques}

Les collisionneurs hadroniques utilisent des hadrons, qui sont des 
particules complexes composées de 3 quarks et de gluons\footnote{Gluon : boson médiateur de l'interaction forte qui maintiennent les quarks ensembles.}. 
Par exemple au LHC\footnote{LHC : Large Hadron Collider, CERN}, on utilise des protons, composés de 2 quarks up et d'1 quark down. 

Comme il s'agit de particules composites, ce sont pas les protons qui collisionnent directement mais ces constituants, appelés partons. 
Chacun porte une fraction indéterminée de l'énergie du proton. 
On ignore donc l'énergie de la collision en amont, il s'agit d'un paramètre libre. 

C'est pourquoi, ils sont utiles pour la découverte de nouvelles particules de masse inconnue, puisqu'ils permettent de balayer tout le spectre de masse sous la gamme d'énergie du collisionneur (au LHC < 14 $\TeV$)\footnote{D'où l'intérêt de nouveaux collisionneurs à des énergies plus élevées et donc des masses de particules produites plus lourdes.}.

\subsection{Collisionneurs leptoniques}

En revanche, les collisionneurs leptoniques utilisent des leptons, qui sont des particules élémentaires. 
Comme le LEP\footnote{LEP : Large Electron-Positron, le prédécesseur du LHC, même tunnel}, qui collisionnait des électrons et des positrons \cite{cern:lep}.

Cette fois-ci, chaque lepton qui collisionne, possède une énergie complète donc connue. 
Puisqu'on leur impulse une énergie précise, ainsi on augmente la statistique pour un certain niveau d'énergie.
Ces collisionneurs sont donc utilisés pour la recherche de précision.\\

Les prochaines générations de collisionneurs, comme ILC, CEPC, CLIC et FCC, ce sont des collisionneurs leptoniques. 
Leur objectif est de préciser les données déjà obtenues, notamment sur le boson de Higgs découvert en 2012 par le LHC\footnote{Higgs : était la pièce manquante du modèle standard des particules, car il permet aux particules d'acquérir une masse}.

\section{Physique du boson de Higgs}

\subsection{Production du boson de Higgs}

Concrètement, on ne mesure pas directement le boson de Higgs mais ses produits de désintégrations sous la forme de jet.
Ainsi on cherche à améliorer la résolutions en énergie de ces jets que l'on détecte \cite{liu:tel-03405418}.

\subsection{Détecteur}

En physique des particules, on utilise des détecteurs appelés calorimètres pour mesurer l'énergie des particules. 
Cette énergie va être déposer par ionisation dans le matériau le long de la trajectoire des particules qui le traverse. 
Il faut donc des algorithmes de reconstruction pour déduire les énergies, les types de particules et les trajectoires.

Pour cela, on utilise des calorimètres à grande granularité qui permet une très bonne performance des Algorithmes de Flux de Particules (PFA) \cite{liu:tel-03405418}. \\

C'est dans ce cadre que la collaboration internationale CALICE, à développer le premier prototype de la famille de calorimètre granulaire SDHCAL, pour Semi-Digital Hadronic CALorimeter, qui a été développé en grande partie à l'IP2I dans l'équipe CMS, auquel j'appartiens pour ce stage.

\subsection{Collisions}

Au cours, de ce stage, je me suis concentrée sur les collisions de type \texttt{nnh} pour neutrino-neutrino-higgs.

Dont voici les diagrammes de Feynman :

\begin{figure}[h!]
	\centering
	\begin{tikzpicture}
	
	\end{tikzpicture}
	\caption{Diagramme de Feynmann de collision $ e^{+} e^{-} \longrightarrow \nu \nu h $}
	\label{Feynman}
\end{figure}

\section{Présentation \& Objectif du Stage}

Pour ce stage, j'ai récupéré les codes de Guillaume \textsc{Garillot}, qui les a développé en 2021 au cours de son post-doctorat à l'IP2I. 
Ils sont en libre accès à l'adresse \url{https://github.com/ggarillot/nnhAnalysis/tree/refactor}.\\

Ce programme \nnhAnalysis permet l'étude de fichiers \SLCIO pour la collision :
\begin{equation}
	\positron \electron \longrightarrow \nnh
\end{equation}

Et l'analyse des canaux de désintégration :

\begin{align}
	h &\longrightarrow W \Wstar \longrightarrow qqqq \\
	h &\longrightarrow b \bbar 
\end{align}

Pour cela, il a utilisé les suites logiciels de \iLCSoft, \url{https://github.com/iLCSoft} (plus précisément \LCIO et \Marlin), qui sont les anciennes suites logicielles.
Mais les nouveaux projets de collisionneurs utiliseront \texttt{Key4HEP} et \Gaudi.\\

Mon objectif est double. Dans un premier temps, comprendre et optimiser les codes existants, \cad le programme \nnhAnalysis de Guillaume \textsc{Garillot} qui utilise \LCIO, \Marlin. 
Puis, je vais transformer son programme pour qu'il puisque correspondre aux nouvelles normes des collisionneurs leptoniques, \texttt{Key4HEP} et \Gaudi.