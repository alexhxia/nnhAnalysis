\chapter{Introduction}

\section{Physique des collisionneurs}

\subsection{Collisionneurs hadroniques}

Collisionneurs pour la découverte

\subsection{Collisionneurs leptoniques}

Collisionneurs de précision

Les prochaines générations de collisionneurs leptoniques sont ILC, CEPC, CLIC et FCC. 
L'objectif étant une meilleurs résolutions de l'énergie des jets \cite{liu:tel-03405418}.

Pour cela, on utilise ici des calorimètres à grande granularité qui permet une très bonne performance des Algorithmes de Flux de Particules (PFA) \cite{liu:tel-03405418}.


\section{SDHCAL (Semi-Digital Hadronic CALorimeter)}

Dans le cadre de la collaboration internationale CALICE, le premier prototype de la famille de calorimètres granulaires SDHCAL a été développer au grande partie à l'IP2I.

tests en Septembre

\subsection{Collisions}
Au cours, de ce stage, je me concentrerais sur les collisions de type \texttt{nnh} pour neutrino-neutrino-higgs

%\section{iLCSoft}

%\section{FCC}

\section{Présentation \& Objectif du Stage}

analyse des canaux :
\begin{equation}
	e^{+} e^{-} \longrightarrow \nu \nu h \left(h \longrightarrow WW \longrightarrow qqqq \right)
\end{equation}

\begin{equation}
	e^{+} e^{-} \longrightarrow \nu \nu h \left(h \longrightarrow b \bbar \right)
\end{equation}