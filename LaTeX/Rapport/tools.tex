\chapter{Outils Numériques}

\section{\texttt{nnhScript}}
\url{https://github.com/alexhxia/nnhAnalysis/tree/main/nnhScript}

\section{\texttt{nnhTest}}

Pour tester les programmes générer avec \texttt{nnhProgram}, j'ai développé 4 programmes en \texttt{python} : 

\begin{figure}[h!]
	\center
	\begin{tabular}{| c | c | c |}
		\hline
			\texttt{•} & \texttt{Processus} & \texttt{Analysis} \\
		\hline
			\texttt{Completed} & \texttt{testProcessorCompleted.py} & \texttt{testAnalysisCompleted.py} \\
		\hline
			\texttt{Same} & \texttt{testProcessorSame.py} & \texttt{testAnalysisSame.py} \\
		\hline
	\end{tabular}
	\caption{Tableau récapitulatif des fonctions de tests}
\end{figure}

\subsection{Programmes \texttt{testXxCompleted.py}}

L'objectif de ce type de programme est de tester si tous les fichiers ont été généré.

\subsubsection{Programmes \texttt{testProcessorCompleted.py}}

Le processus est complet si tous les dossiers de 

\subsubsection{Programmes \texttt{testAnalysisCompleted.py}}


\subsection{Programmes \texttt{testXxSame.py}}

\subsubsection{Programmes \texttt{testProcessorSame.py}}

\subsubsection{Programmes \texttt{testAnalysisSame.py}}



\url{https://github.com/alexhxia/nnhAnalysis/tree/main/nnhTest}