\documentclass[9pt]{beamer}

\usetheme{Pittsburgh}

\usepackage[utf8]{inputenc}
\usepackage[french]{babel}
\usepackage[T1]{fontenc}
\usepackage{amsmath}
\usepackage{amsfonts}
\usepackage{amssymb}
\usepackage{graphicx}
\usepackage{hyperref}
\usepackage{url}
\usepackage{tikz}
\usepackage{tabularx}

\author[Hocine]{Alexia HOCINE}

\title{
	Comparaison des prédictions des suites logicielles de ILC (iLCSoft) et de FCC (key4HEP) \\
	sur un signal $ e^{+} e^{-} \longrightarrow Z H $ 
}
\subtitle{Gérald Grenier (IP2I équipe CMS, FCC, SDHCAL)}

\titlegraphic{
	\includegraphics[width=0.18\textwidth]{../img/UdL-logo.png}
    \hspace{5cm}
    \includegraphics[width=0.15\textwidth]{../img/Logo_IP2I.png}
}

\setbeamercovered{transparent} 
\setbeamertemplate{navigation symbols}{} 
\logo{} 
\institute[UCBL]{Université de Claude Bernard Lyon 1} 
\date[2022]{Juin 2022} 
\subject{nnhA} 

% Décompte des pages sans les annexes
\setbeamertemplate{caption}[numbered]

\newcommand{\backupbegin}{
   \newcounter{framenumberappendix}
   \setcounter{framenumberappendix}{\value{framenumber}}
}

\newcommand{\backupend}{
   \addtocounter{framenumberappendix}{-\value{framenumber}}
   \addtocounter{framenumber}{\value{framenumberappendix}} 
}


% math 
\newcommand{\bbar}{\overline{b}}

\begin{document}

%%%%%%%%%%%%%%%%%%%%%%%%%%%%%%%%%%%%%%%%%%%%%%%%%%%%%%%%%%%%%%%%%%%%%%%%%%%%%%

\begin{frame}
	\titlepage
\end{frame}

%%%%%%%%%%%%%%%%%%%%%%%%%%%%%%%%%%%%%%%%%%%%%%%%%%%%%%%%%%%%%%%%%%%%%%%%%%%%%%
\section{Physique analysée}

\begin{frame}{Introduction}

	$$ e^{+} e^{-} \longrightarrow \nu \nu h $$

	\begin{columns}
		
		\begin{column}{0.55\textwidth}
			\begin{description}
				\item[$e^{+} e^{-}$] collisionneur leptonique
				\item[$\nu \nu h$] neutrinos-neutrinos-higgs
			\end{description}
		\end{column}
		
		\begin{column}{0.4\textwidth}
			\begin{block}{Canaux à analyser}
				\begin{enumerate}
					\item $h \longrightarrow WW \longrightarrow qqqq$
					\item $h \longrightarrow b\bbar$
				\end{enumerate}
			\end{block}
		\end{column}
	\end{columns}

\end{frame}

%%%%%%%%%%%%%%%%%%%%%%%%%%%%%%%%%%%%%%%%%%%%%%%%%%%%%%%%%%%%%%%%%%%%%%%%%%%%%%

\begin{frame}{Sommaire}
	\tableofcontents
\end{frame}

%%%%%%%%%%%%%%%%%%%%%%%%%%%%%%%%%%%%%%%%%%%%%%%%%%%%%%%%%%%%%%%%%%%%%%%%%%%%%%
\section{Programmes}

\subsection{original}

\begin{frame}{original}{https://github.com/alexhxia/nnhAnalysis/tree/main/nnhProgram/original}

\begin{block}{Projet initial}
	\url{https://github.com/ggarillot/nnhAnalysis/tree/refactor}
\end{block}

\begin{columns}

	\begin{column}{0.33\textwidth}
		\begin{block}{\texttt{miniDSTMaker}}
			Télécharge du server \texttt{lyogrid06} les fichiers \texttt{DST} de \texttt{DESY} : \texttt{.lcio}
		\end{block}
	\end{column}
	
	\begin{column}{0.33\textwidth}
		\begin{block}{\texttt{processor}}
			Transforme les fichiers \texttt{.lcio} en \texttt{.root} par type de processus
		\end{block}
	\end{column}
	
	\begin{column}{0.33\textwidth}
		\begin{block}{\texttt{analysis}}
			Entraine une BDT, pour obtenir l'analyse statistique des évènements
		\end{block}
	\end{column}

\end{columns}

\begin{block}{Type de processus}
	\begin{figure}
		\center
		\includegraphics[width=\textwidth]{../img/listeProcessus.png} 
	\end{figure}
\end{block}

\end{frame}

%%%%%%%%%%%%%%%%%%%%%%%%%%%%%%%%%%%%%%%%%%%%%%%%%%%%%%%%%%%%%%%%%%%%%%%%%%%%%%
\subsection{iLCSoft}

\begin{frame}{iLCSoft}{https://github.com/alexhxia/nnhAnalysis/tree/main/nnhProgram/ilcsoft}

	\begin{block}{Améliorations apporter à \texttt{processor} et \texttt{analysis}}
		\begin{itemize}
			\item Réécritures minimes (typographie, typage \texttt{auto})
			\item Modification des noms de certaines fonctions
			\item Ajouts de commentaires (clarification des contrats)
			\item Nouvelle classe pour simplifier l'utilisation des codes PDG : \texttt{PDGInfo.XX}\\
					\texttt{XX = \{hh, cc\}}
			\item Réorganisation de la gestion des fichiers des sortis pour permettre l'exécution en parallèle 
		\end{itemize}
	\end{block}
	
	\begin{block}{\texttt{miniDSTMaker}}
		\begin{itemize}
			\item Non pertinent pour ce stage, puisque les données sont locales
		\end{itemize}
	\end{block}

\end{frame}

%%%%%%%%%%%%%%%%%%%%%%%%%%%%%%%%%%%%%%%%%%%%%%%%%%%%%%%%%%%%%%%%%%%%%%%%%%%%%%
\subsection{FCC}

\begin{frame}{FCC}{https://github.com/alexhxia/nnhAnalysis/tree/main/nnhProgram}

	\begin{block}{Ajout du programme en amont : \texttt{gaudi}}
		\begin{itemize}
			\item Tranforme les fichiers \texttt{.lcio} en fichier exploitable par iLCSoft en fichier exploitable pour FCC.
			\item De \texttt{EDM4HEP} vers \texttt{FCCEDM}
		\end{itemize}
	\end{block}

	\begin{block}{\texttt{processor}}
		\begin{itemize}
			\item Change toutes les utilisations de la suite logicielle d'iLCSoft vers key4HEP
		\end{itemize}
	\end{block}

	\begin{block}{\texttt{analysis}}
		\begin{itemize}
			\item Ne demande aucune modification
		\end{itemize}
	\end{block}

\end{frame}

%%%%%%%%%%%%%%%%%%%%%%%%%%%%%%%%%%%%%%%%%%%%%%%%%%%%%%%%%%%%%%%%%%%%%%%%%%%%%%
\section{Outils de Numérique}

\subsection{Script}

\begin{frame}{Outils de Numérique : \texttt{Script}}{https://github.com/alexhxia/nnhAnalysis/tree/main/nnhScript}

\begin{block}{Liste de nouveaux scripts}
	\begin{description}
		\item[\texttt{nnh}] programme général 
		\begin{itemize}
			\item permet de choisir :
			\begin{itemize}
				\item combien de programme \texttt{processus} et \texttt{analysis} on souhaite
			\end{itemize}
		\end{itemize}
		\begin{description}
			\item[\texttt{nnhProcessor}] lance un programme \texttt{processus} complet
			\item[\texttt{nnhAnalysis}] lance un programme \texttt{analysis} complet
			\begin{description}
				\item[\texttt{prepareBDT}] lance le programme \texttt{prepareBDT}
				\item[\texttt{launchBDT}] lance le programme \texttt{launchBDT}
			\end{description}
		\end{description}
	\end{description}
\end{block}

\end{frame}

%%%%%%%%%%%%%%%%%%%%%%%%%%%%%%%%%%%%%%%%%%%%%%%%%%%%%%%%%%%%%%%%%%%%%%%%%%%%%%
\subsection{Test}

\begin{frame}{Outils de Numérique : \texttt{Test}}{https://github.com/alexhxia/nnhAnalysis/tree/main/nnhTest}

\begin{block}{Programme de tests : \texttt{testXxYy.py}}
	\begin{itemize}
		\item Teste grâce à le fonction de Kolmogorov - développé par ROOT (CERN)
		\item Teste les fichiers de sortis :
		\begin{itemize}
			\item des programmes \texttt{Xx = \{processus, analysis\}}
			\item de type \texttt{Yy = \{Completed, Same\}}
		\end{itemize}
	\end{itemize}
	\begin{tabular}{| c | c | c |}
		\hline
			\texttt{•} & \texttt{Processus} & \texttt{Analysis} \\
		\hline
			\texttt{Completed} & \texttt{testProcessorCompleted.py} & \texttt{testAnalysisCompleted.py} \\
		\hline
			\texttt{Same} & \texttt{testProcessorSame.py} & \texttt{testAnalysisSame.py} \\
		\hline
	\end{tabular}
	\begin{description}
		\item[\texttt{Completed}] teste si tous les fichiers ont bien été générer
		\item[\texttt{Same}]	teste les différences entre 2 fichiers
		\begin{description}
			\item[\texttt{processus}] tous les fichiers sont sensés être identiques
			\item[\texttt{analysis}]	tous les fichiers sont sensés être équivalents
		\end{description}
	\end{description}
\end{block}

\end{frame}

%%%%%%%%%%%%%%%%%%%%%%%%%%%%%%%%%%%%%%%%%%%%%%%%%%%%%%%%%%%%%%%%%%%%%%%%%%%%%%
\section{Conclusion}

\begin{frame}{Conclusion}

\begin{columns}

	\begin{column}{0.45\textwidth}
		\begin{block}{Travails réalisés}
			\begin{itemize}
		 		\item Optimisation des codes pour iLCSoft
		 		\item Adaptation au projet FCC
		 		\item Automatisation des programmes
		 		\item Programmation de codes de test
			\end{itemize}
		\end{block}
	\end{column}
	
	\begin{column}{0.45\textwidth}
		\begin{block}{Compétences}
			\begin{itemize}
		 		\item Programmation C++, ROOT, Python, Script bash
		 		\item Utilisation de BDT
		 		\item Statistiques
			\end{itemize}
		\end{block}
	\end{column}

\end{columns}

\end{frame}

%%%%%%%%%%%%%%%%%%%%%%%%%%%%%%%%%%%%%%%%%%%%%%%%%%%%%%%%%%%%%%%%%%%%%%%%%%%%%%
\backupbegin
%%%%%%%%%%%%%%%%%%%%%%%%%%%%%%%%%%%%%%%%%%%%%%%%%%%%%%%%%%%%%%%%%%%%%%%%%%%%%%

\begin{frame}{Sources des Figures}

\listoftables 

%\begin{description}
	%\item[\ref{fig:Michelson}] \url{public.virgo-gw.eu/index.php?gmedia=Rv7bB&t=g\#photoBox-54}
%\end{description}

\end{frame}

%%%%%%%%%%%%%%%%%%%%%%%%%%%%%%%%%%%%%%%%%%%%%%%%%%%%%%%%%%%%%%%%%%%%%%%%%%%%%%

\begin{frame}{Bibliographie}

%\begin{description}
	%\item[\ref{fig:Michelson}] \url{public.virgo-gw.eu/index.php?gmedia=Rv7bB&t=g\#photoBox-54}
%\end{description}

\end{frame}

%%%%%%%%%%%%%%%%%%%%%%%%%%%%%%%%%%%%%%%%%%%%%%%%%%%%%%%%%%%%%%%%%%%%%%%%%%%%%%
\appendix
\begin{frame}

\begin{center}
\Huge{Suppléments}
\end{center}

\end{frame}

%%%%%%%%%%%%%%%%%%%%%%%%%%%%%%%%%%%%%%%%%%%%%%%%%%%%%%%%%%%%%%%%%%%%%%%%%%%%%%

\begin{frame}{Annexes}
\end{frame}

%%%%%%%%%%%%%%%%%%%%%%%%%%%%%%%%%%%%%%%%%%%%%%%%%%%%%%%%%%%%%%%%%%%%%%%%%%%%%%
\backupend
\end{document}
